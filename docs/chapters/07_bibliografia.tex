% ============================================================
% CHAPTER: BIBLIOGRAPHY
% Autor: Eloi Egea Rada
% Tema: Complete bibliography and references
% ============================================================

\chapter{Bibliography}

% LaTeX Bibliography Integration
% This chapter manages all references used throughout this TFG.
% References are formatted according to IEEE style.
% For technical details on IEEE citation format, see:
% - IEEE Editorial Style Manual: https://ieeeauthorcenter.ieee.org/
% - IEEE Reference Examples: https://ieee-dataport.org/sites/default/files/analysis/27/IEEE%20Citation%20Guidelines.pdf

% The bibliography is generated from references.bib using BibTeX
% Compilation sequence: pdflatex -> bibtex -> pdflatex -> pdflatex

\nocite{*}  % Include all entries from references.bib (optional - remove if using specific citations)

% Bibliography style (IEEE format preferred for engineering/CS)
\bibliographystyle{IEEE}

% Input bibliography from references.bib file
\bibliography{references}

\newpage

\section{Bibliography Notes}

This bibliography compiles all sources referenced throughout this Final Degree Project (TFG).
References are organized using the IEEE citation style, which is the standard for computer science
and engineering publications. The complete reference database is maintained in \texttt{references.bib}.

\subsection{Reference Categories}

The bibliography includes references across the following categories:

\begin{itemize}

\item \textbf{Cybersecurity Frameworks and Standards}
\begin{itemize}
\item NIST Cybersecurity Framework
\item OWASP (Open Web Application Security Project) guidelines
\item CIS Critical Controls and benchmarks
\item PTES (Penetration Testing Execution Standard)
\item CEH (Certified Ethical Hacker) framework documentation
\end{itemize}

\item \textbf{Penetration Testing and Ethical Hacking}
\begin{itemize}
\item Penetration testing methodologies
\item Vulnerability assessment techniques
\item Exploitation frameworks and tools
\item Post-exploitation and privilege escalation
\item Network security and defense mechanisms
\end{itemize}

\item \textbf{Cybersecurity Education and Pedagogy}
\begin{itemize}
\item Hands-on learning and laboratory design
\item Security education effectiveness research
\item Cybersecurity curriculum development
\item Active learning strategies
\item Assessment methodologies for technical competencies
\end{itemize}

\item \textbf{Virtualization and Laboratory Platforms}
\begin{itemize}
\item EVE-NG (Emulated Virtual Environment - Next Generation)
\item KVM and Linux virtualization
\item Network simulation and emulation
\item Alternative platforms: HackTheBox, TryHackMe, Cisco Academy, SEED Labs, GOAD
\end{itemize}

\item \textbf{Penetration Testing Tools and Technologies}
\begin{itemize}
\item Kali Linux and security tool collections
\item Metasploit Framework and exploitation
\item Burp Suite and web application testing
\item Nmap and network reconnaissance
\item Wireshark and network analysis
\item OWASP ZAP and automated scanning
\end{itemize}

\item \textbf{Academic and Industry Publications}
\begin{itemize}
\item IEEE Xplore publications on cybersecurity
\item ACM Digital Library research papers
\item Conference proceedings (InfoSecCD, ASEE, etc.)
\item Industry white papers and technical reports
\item Case studies of security incidents and lessons learned
\end{itemize}

\end{itemize}

\subsection{How to Use This Bibliography}

\begin{enumerate}

\item \textbf{Citation in Text}: Use \verb|\cite{key}| where \texttt{key} is the BibTeX identifier
\begin{verbatim}
Example: According to the NIST Cybersecurity Framework \cite{NIST_CSF_2018},
organizations must implement controls across five core functions.
\end{verbatim}

\item \textbf{Multiple References}: Use multiple identifiers separated by commas
\begin{verbatim}
Example: Several frameworks address cybersecurity governance \cite{NIST_CSF_2018,CIS_Controls_2018,ISO_27001}.
\end{verbatim}

\item \textbf{Reference Format Verification}: All references maintain IEEE format:
\begin{itemize}
\item Author(s) names in order
\item Title of work in quotes for articles, italics for books
\item Publication details (journal, conference, publisher)
\item Date and page numbers
\item DOI or URL where available
\end{itemize}

\end{enumerate}

\subsection{Accessing Referenced Materials}

Many of the referenced materials are available through the following channels:

\begin{itemize}

\item \textbf{NIST Publications}: https://www.nist.gov/publications
\item \textbf{OWASP Resources}: https://owasp.org/
\item \textbf{CIS Benchmarks}: https://www.cisecurity.org/cis-benchmarks/
\item \textbf{IEEE Xplore}: https://ieeexplore.ieee.org/
\item \textbf{ACM Digital Library}: https://dl.acm.org/
\item \textbf{GitHub}: https://github.com/ (for open-source tools and documentation)
\item \textbf{Official Tool Documentation}: Specific URLs maintained in references.bib

\end{itemize}