% ============================================================
% Autor: Eloi Egea Rada
% Tema: Complete specification of system, technological, and quality requirements
% ============================================================

\chapter{Functional and Non-Functional Requirements}

%\section{Introduction}

This chapter specifies the complete requirements for the EVE-NG-based cybersecurity teaching package, encompassing functional requirements (RF) that define what the system must do, technological requirements (TR) that specify the technical environment, and non-functional requirements (NFR) that address quality attributes and compliance standards.

\section{Functional Requirements}

Functional requirements define the core capabilities and behaviors expected from the teaching package.

\subsection{RF1: Design and Implement EVE-NG Topologies}

\textbf{Description}: Create four separate, functional, and interconnected network topologies in EVE-NG format (.unl files) that simulate real-world cybersecurity scenarios.

\textbf{Detailed Requirements}:

\begin{itemize}

\item \textbf{RF1.1}: Design 4 distinct network topologies, each targeting a specific pentesting phase:
\begin{itemize}
\item Lab 1: Reconnaissance topology (passive intelligence gathering)
\item Lab 2: Web application security topology (active vulnerability discovery)
\item Lab 3: Network analysis topology (lateral movement and monitoring)
\item Lab 4: Privilege escalation topology (post-exploitation scenarios)
\end{itemize}

\item \textbf{RF1.2}: Each topology must include:
\begin{itemize}
\item Minimum 5-8 virtual machines per lab
\item Realistic network architecture (DMZ, internal networks, monitoring zones)
\item Vulnerable services configured per OWASP and NIST standards
\item Network segmentation with VLANs where appropriate
\end{itemize}

\item \textbf{RF1.3}: EVE-NG compatibility requirements:
\begin{itemize}
\item .unl file format compatible with EVE-NG Community Edition
\item Additional compatibility with EVE-NG Professional Edition
\item KVM hypervisor support (Ubuntu 20.04+ LTS)
\item Minimum host specification: 16GB RAM, 200GB storage
\end{itemize}

\item \textbf{RF1.4}: Topology documentation:
\begin{itemize}
\item Network diagram for each lab (logical and physical views)
\item IP addressing scheme documented
\item Service inventory and port mappings
\item Attack surface identification and justification
\end{itemize}

\end{itemize}

\textbf{Success Criteria}:
\begin{itemize}
\item All 4 .unl files deploy successfully without errors
\item VM boot success rate greater or equal to 95 percent
\item Network connectivity verified between all nodes
\item Estimated deployment time: less or equal to 2 minutes per lab
\end{itemize}

\subsection{RF2: Implement Vulnerabilities and Attack Scenarios}

\textbf{Description}: Implement realistic vulnerabilities and attack scenarios aligned with industry frameworks (PTES, OWASP, NIST, CEH).

\textbf{Detailed Requirements}:

\begin{itemize}

\item \textbf{RF2.1}: Lab 1 - Reconnaissance (PTES Phase 1)
\begin{itemize}
\item Passive reconnaissance tools (Shodan, DNS enumeration)
\item Active scanning tools (Nmap, banner grabbing)
\item Service discovery scenarios
\item Information gathering reporting
\end{itemize}

\item \textbf{RF2.2}: Lab 2 - Web Vulnerabilities (OWASP Top 10)
\begin{itemize}
\item Injection vulnerabilities (SQL, command injection)
\item Authentication bypasses
\item Session management flaws
\item Cross-Site Scripting (XSS) scenarios
\item Cross-Site Request Forgery (CSRF) vulnerabilities
\item Broken Access Control
\item Security misconfiguration examples
\item Sensitive data exposure scenarios
\end{itemize}

\item \textbf{RF2.3}: Lab 3 - Network Analysis (NIST Standards)
\begin{itemize}
\item Network segmentation analysis
\item Lateral movement techniques
\item Man-in-the-Middle (MITM) attack scenarios
\item Network traffic analysis with Wireshark
\item IDS/IPS evasion scenarios
\item VPN and encryption bypass (where applicable)
\end{itemize}

\item \textbf{RF2.4}: Lab 4 - Privilege Escalation (CEH Framework)
\begin{itemize}
\item Local privilege escalation vulnerabilities
\item Kernel exploits (outdated systems)
\item Weak file permissions and sudo misconfigurations
\item Windows privilege escalation (UAC bypass, token impersonation)
\item Post-exploitation persistence mechanisms
\end{itemize}

\item \textbf{RF2.5}: Vulnerability implementation standards:
\begin{itemize}
\item All vulnerabilities must be fixable by administrators
\item No permanently broken or unrecoverable systems
\item Clear reset procedure for each vulnerability state
\item Automated scanning tools must successfully identify vulnerabilities
\end{itemize}

\end{itemize}

\textbf{Success Criteria}:
\begin{itemize}
\item Minimum 3-4 vulnerability scenarios per lab
\item 100 percent of implemented vulnerabilities exploitable through documented methods
\item Average exploitation time: 15-45 minutes per vulnerability
\item All vulnerabilities remediable with documented procedures
\end{itemize}

\subsection{RF3: Create Structured Assessment Materials}

\textbf{Description}: Develop comprehensive assessment rubrics, learning objectives, and evaluation materials aligned with GEISI curriculum.

\textbf{Detailed Requirements}:

\begin{itemize}

\item \textbf{RF3.1}: Learning outcomes definition:
\begin{itemize}
\item 3-4 specific learning outcomes per lab
\item Alignment with BLOOM's taxonomy (knowledge, application, analysis)
\item Clear connection to GEISI degree competencies
\end{itemize}

\item \textbf{RF3.2}: Assessment rubrics:
\begin{itemize}
\item Detailed rubrics for each lab (minimum 15 pages per lab)
\item Quantifiable evaluation criteria
\item Point allocation per criterion (clear grading scale)
\item Example student responses (exemplars)
\end{itemize}

\item \textbf{RF3.3}: Support materials:
\begin{itemize}
\item Student lab manuals with step-by-step guidance
\item Teacher guides with deployment and customization instructions
\item Expected outcomes and common pitfalls documentation
\item Troubleshooting guides for common technical issues
\end{itemize}

\item \textbf{RF3.4}: Validation materials:
\begin{itemize}
\item Automated validation scripts to verify successful lab completion
\item Manual verification procedures for educators
\item Documentation of success indicators
\end{itemize}

\end{itemize}

\textbf{Success Criteria}:
\begin{itemize}
\item 4 complete assessment rubrics (1 per lab)
\item 4 student manuals (minimum 15 pages each)
\item 1 comprehensive teacher guide
\item 100 percent of features documented with examples
\end{itemize}

\section{Technological Requirements}

Technological requirements specify the infrastructure, tools, and platforms required for system implementation and deployment.

\subsection{TR1: Virtualization Infrastructure}

\begin{itemize}

\item \textbf{TR1.1}: EVE-NG Platform
\begin{itemize}
\item Version: EVE-NG 5.0 or later
\item Editions supported: Community (free) and Professional
\item Deployment: Docker container or bare-metal installation
\item Web interface: Accessible via HTTPS on port 443
\item API: RESTful API for automation and integration
\end{itemize}

\item \textbf{TR1.2}: Hypervisor
\begin{itemize}
\item Primary: KVM (Kernel-based Virtual Machine)
\item Host OS: Ubuntu Server 20.04 LTS or 22.04 LTS
\item QEMU integration for VM management
\item Nested virtualization support for advanced scenarios
\end{itemize}

\item \textbf{TR1.3}: Host System Specifications
\begin{itemize}
\item CPU: 2x Intel Xeon E5-2600 v3 series or equivalent (minimum 16 cores)
\item RAM: 64GB minimum (recommended 128GB for simultaneous labs)
\item Storage: 2TB+ NVMe or SSD (minimum 50GB per VM image)
\item Network: Gigabit Ethernet (10Gbps preferred)
\item Power redundancy: UPS for service continuity
\end{itemize}

\item \textbf{TR1.4}: Network Configuration
\begin{itemize}
\item Layer 2 switching capabilities within EVE-NG
\item VirtualBox-based network simulation where needed
\item VLAN support for network segmentation scenarios
\item IPv4 and IPv6 support
\end{itemize}

\end{itemize}

\subsection{TR2: Security Tools and Penetration Testing Framework}

\begin{itemize}

\item \textbf{TR2.1}: Attacking Platform
\begin{itemize}
\item Primary: Kali Linux 2024.x
\item Alternative: Parrot Security OS
\item Installation method: EVE-NG pre-built image or custom import
\item Tools package: Full toolkit including reconnaissance, scanning, and exploitation
\end{itemize}

\item \textbf{TR2.2}: Core Security Tools (installed on Kali)
\begin{itemize}
\item \textbf{Reconnaissance}: Nmap, Shodan CLI, Amass, Sublist3r
\item \textbf{Scanning}: Nessus (optional), OpenVAS, Nikto
\item \textbf{Exploitation}: Metasploit Framework 6.x, ExploitDB, Searchsploit
\item \textbf{Web Testing}: Burp Suite Community/Professional, OWASP ZAP
\item \textbf{Network Analysis}: Wireshark, tcpdump, mitmproxy
\item \textbf{Post-Exploitation}: Mimikatz, Rubeus, BloodHound
\item \textbf{Programming}: Python 3.10+, Bash scripting
\end{itemize}

\item \textbf{TR2.3}: Target System Images
\begin{itemize}
\item Ubuntu Server 18.04/20.04 LTS (Linux targets)
\item CentOS 7/8 (enterprise Linux)
\item Windows Server 2016/2019 (Windows targets)
\item Vulnerable applications: DVWA, WebGoat, Juice Shop, Mutillidae
\end{itemize}

\item \textbf{TR2.4}: Compliance and Documentation
\begin{itemize}
\item All tools must be freely available or have educational licenses
\item Open-source preference for long-term sustainability
\item Tool version specifications documented
\item Legal and ethical use guidelines included
\end{itemize}

\end{itemize}

\subsection{TR3: Scripting and Automation Framework}

\begin{itemize}

\item \textbf{TR3.1}: Languages
\begin{itemize}
\item Bash scripting for system automation (deployment, reset, validation)
\item Python 3.10+ for complex automation and integration
\item Optional: Go or Rust for performance-critical components
\end{itemize}

\item \textbf{TR3.2}: Automation Scripts
\begin{itemize}
\item Deploy scripts: Initialize lab topologies and configurations
\item Reset scripts: Restore lab to initial vulnerable state
\item Validation scripts: Verify security effectiveness and readiness
\item Monitoring scripts: Track lab health and performance
\end{itemize}

\item \textbf{TR3.3}: Integration Points
\begin{itemize}
\item EVE-NG API integration for programmatic lab management
\item Webhook support for event-driven automation
\item Container orchestration (Docker) for microservices
\item Version control integration (Git, GitHub)
\end{itemize}

\end{itemize}

\section{Non-Functional Requirements}

Non-functional requirements specify quality attributes and operational characteristics.

\subsection{NFR1: Performance and Efficiency}

\begin{itemize}

\item \textbf{NFR1.1}: Deployment Performance
\begin{itemize}
\item Lab deployment time: less or equal to 2 minutes from topology load to ready state
\item VM boot time: less or equal to 90 seconds per VM (95 percent of cases)
\item Reset operation: less or equal to 30 seconds to restore to initial state
\item Concurrent lab instances: Support minimum 5 simultaneous labs
\end{itemize}

\item \textbf{NFR1.2}: Resource Optimization
\begin{itemize}
\item VM image sizes: less or equal to 5GB per image (compressed)
\item Memory footprint: less or equal to 6GB per lab instance during execution
\item Network bandwidth: Minimal external dependencies (all local simulation)
\end{itemize}

\item \textbf{NFR1.3}: Scalability
\begin{itemize}
\item Support for 20+ simultaneous students (1 lab per student minimum)
\item Support for 30+ concurrent educational deployments
\item Horizontal scaling through distributed EVE-NG nodes
\end{itemize}

\end{itemize}

\subsection{NFR2: Compliance with Security Frameworks}

\begin{itemize}

\item \textbf{NFR2.1}: NIST Cybersecurity Framework Alignment
\begin{itemize}
\item Core Functions: Identify, Protect, Detect, Respond, Recover
\item Implementation mapped to lab exercises
\item Assessment procedures aligned with NIST guidelines
\end{itemize}

\item \textbf{NFR2.2}: CIS Critical Controls Coverage
\begin{itemize}
\item Controls 1-18 incorporated where pedagogically appropriate
\item Assessment rubrics aligned with CIS benchmarks
\item Remediation procedures follow CIS recommendations
\end{itemize}

\item \textbf{NFR2.3}: OWASP Standards Integration
\begin{itemize}
\item OWASP Top 10 vulnerabilities fully covered
\item OWASP Testing Guide methodology integrated
\item OWASP Development Practices reflected in lab design
\end{itemize}

\item \textbf{NFR2.4}: Penetration Testing Standards
\begin{itemize}
\item PTES (Penetration Testing Execution Standard) phases mapped to labs
\item CEH (Certified Ethical Hacker) curriculum alignment
\item Ethical hacking principles strictly enforced
\end{itemize}

\end{itemize}

\subsection{NFR3: Reliability and Availability}

\begin{itemize}

\item \textbf{NFR3.1}: System Reliability
\begin{itemize}
\item Mean Time Between Failures (MTBF): greater or equal to 100 hours continuous operation
\item Mean Time To Recovery (MTTR): less or equal to 5 minutes
\item Scenario reproducibility: 100 percent (deterministic behavior)
\item Bug severity: Zero critical bugs at release
\end{itemize}

\item \textbf{NFR3.2}: Data Integrity
\begin{itemize}
\item No unintended data loss during lab operations
\item Snapshot and restore functionality for state management
\item Backup procedures for sensitive configurations
\end{itemize}

\item \textbf{NFR3.3}: Availability
\begin{itemize}
\item Planned maintenance windows: Weekends only
\item System availability: greater or equal to 99 percent during instructional periods
\item Graceful degradation if resources become limited
\end{itemize}

\end{itemize}

\subsection{NFR4: Security and Ethical Compliance}

\begin{itemize}

\item \textbf{NFR4.1}: Educational Sandbox Isolation
\begin{itemize}
\item Complete network isolation from production systems
\item No outbound internet access from lab environments
\item Contained attack scenarios (no external targets)
\end{itemize}

\item \textbf{NFR4.2}: Access Control and Accountability
\begin{itemize}
\item Role-based access control (RBAC) for lab management
\item Audit logs of all student actions (where applicable)
\item Clear identification of authorized vs. unauthorized activities
\end{itemize}

\item \textbf{NFR4.3}: Ethical Guardrails
\begin{itemize}
\item Mandatory ethics training and certification before lab access
\item Written agreements regarding responsible use
\item Monitoring and escalation procedures for suspicious activities
\end{itemize}

\end{itemize}

\subsection{NFR5: Usability and Support}

\begin{itemize}

\item \textbf{NFR5.1}: Ease of Use
\begin{itemize}
\item Average time to deploy a lab: 5-10 minutes for faculty
\item Intuitive web interface with clear navigation
\item Context-sensitive help and tooltips
\item Minimal technical prerequisites for students
\end{itemize}

\item \textbf{NFR5.2}: Documentation and Support
\begin{itemize}
\item Comprehensive user manuals and quick-start guides
\item Video tutorials for complex procedures (2-3 minutes each)
\item FAQ documentation (minimum 50 entries)
\item Responsive support channel (email, discussion forum, chat)
\end{itemize}

\item \textbf{NFR5.3}: Accessibility
\begin{itemize}
\item WCAG 2.1 Level AA compliance for web interfaces
\item Screen reader compatibility
\item Keyboard navigation support
\item Clear contrast ratios for visual elements
\end{itemize}

\end{itemize}

\subsection{NFR6: Institutional Sustainability}

\begin{itemize}

\item \textbf{NFR6.1}: Long-term Maintainability
\begin{itemize}
\item All components open-source or freely available tools
\item No vendor lock-in dependencies
\item Source code in GitHub with clear documentation
\item Backward compatibility with older EVE-NG versions where feasible
\end{itemize}

\item \textbf{NFR6.2}: Curriculum Alignment
\begin{itemize}
\item Explicit mapping to GEISI degree learning outcomes
\item Integration with existing course structures
\item Flexible customization for different instructional contexts
\item Support for multi-level deployment (100/300/400-level courses)
\end{itemize}

\item \textbf{NFR6.3}: Scalability for Institutional Adoption
\begin{itemize}
\item Deployable on institutional infrastructure (not SaaS-dependent)
\item Support for multiple faculty deployments simultaneously
\item Resource reservation and scheduling capabilities
\item Reporting and analytics for institutional assessment
\end{itemize}

\end{itemize}

\section{Requirements Traceability Matrix}

\begin{table}[H]
  \centering
  \small
  \setlength{\tabcolsep}{4pt}
  \renewcommand{\arraystretch}{1.1}
  \begin{tabularx}{\textwidth}{l l X l}
    \toprule
    \textbf{ID} & \textbf{Type} & \textbf{Requirement} & \textbf{Scope} \\
    \midrule
    RF1.1  & Functional   & Design 4 EVE-NG topologies                    & All labs     \\
    RF1.2  & Functional   & 5–8 VMs per topology                          & All labs     \\
    RF2.1  & Functional   & Lab 1 reconnaissance scenarios                & Lab 1        \\
    RF2.2  & Functional   & Lab 2 OWASP vulnerabilities                    & Lab 2        \\
    RF2.3  & Functional   & Lab 3 network analysis                         & Lab 3        \\
    RF2.4  & Functional   & Lab 4 privilege escalation                     & Lab 4        \\
    RF3.1  & Functional   & Learning outcomes defined                      & All labs     \\
    RF3.2  & Functional   & Assessment rubrics                             & All labs     \\
    TR1.1  & Technical    & EVE-NG 5.0+                                    & Platform     \\
    TR1.3  & Technical    & Host 64GB RAM, 2TB storage                     & Infrastructure \\
    TR2.1  & Technical    & Kali Linux 2024 as attacking platform         & Tools        \\
    NFR1.1 & Non-Func.    & Deploy in $\leq$ 2 minutes                     & Performance  \\
    NFR2.1 & Non-Func.    & NIST alignment                                 & Compliance   \\
    NFR3.1 & Non-Func.    & MTBF $\geq$ 100h                               & Reliability  \\
    NFR4.1 & Non-Func.    & Network isolation                              & Security     \\
    NFR6.1 & Non-Func.    & Open-source tools                              & Sustainability \\
    \bottomrule
  \end{tabularx}
  \caption[Requirements traceability]{Requirements traceability matrix for the EVE-NG laboratory platform}
  \label{tab:requirements_traceability}
\end{table}

\begin{comment}
\section{Conclusion}

These comprehensive functional, technological, and non-functional requirements provide the complete specification for the EVE-NG-based cybersecurity teaching package. Adherence to these requirements ensures that the deliverables meet educational objectives, technical quality standards, and institutional sustainability criteria.
\end{comment}