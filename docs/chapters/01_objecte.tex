
% 01_objecte.tex
% Document: TFG Template 
% Idioma: Inglés
\section{Objective of the TFG}
\subsection{Project Description}

The objective of this Final Degree Project is to develop a reusable teaching package of cybersecurity practical labs using the EVE-NG (Emulated Virtual Environment - Next Generation) platform for the course "Introduction to Cybersecurity" of the Bachelor's Degree in Computer Engineering in Management and Information Systems.

The project consists of creating a structured set of four virtualized labs covering the main areas of ethical pentesting: reconnaissance and enumeration, web application vulnerabilities, traffic analysis and cryptography, and privilege escalation. Each lab will include complete network topologies with preconfigured virtual machines, automation scripts for scenario deployment and reset, and comprehensive technical documentation.

\subsection{Rationale and Justification}

\subsection{Educational Context and Identified Needs}

Cybersecurity training requires safe practical environments where students can experiment with pentesting techniques without risks to real systems. Currently, the "Introduction to Cybersecurity" course at Tecnocampus has virtualized infrastructure developed in previous projects, but lacks a structured, automated, and reusable teaching package that facilitates both teaching and autonomous learning.

\subsubsection{Added Value and Innovation}

This TFG adds value in multiple dimensions:

\begin{enumerate}
\item \textbf{Reusability and Scalability}: The teaching package is designed to be used in multiple editions of the course, reducing the preparation load for the teaching staff.
\item \textbf{Automation of the Life Cycle}: Deployment, reset, and validation scripts allow efficient lab management, minimizing time spent on administrative tasks.
\item \textbf{Self-Guided Learning}: Structured documentation and assessment rubrics facilitate autonomous learning for students.
\item \textbf{Professional Standards}: The scenarios reproduce real vulnerabilities and techniques used in the professional field of cybersecurity.
\end{enumerate}

\subsection{Expected Outcomes}

The implementation of this teaching package will enable:

\begin{itemize}
\item Improved student learning experiences through realistic practices
\item Reduced preparation and management time for teachers
\item Enhanced reproducibility and consistency in student assessment
\item Establishment of a solid foundation for future developments in cybersecurity training
\end{itemize}

\subsection{Reference Framework}

\subsubsection{Academic Context}

This project is framed within the field of computer engineering applied to education, combining technical competencies in:

\begin{itemize}
\item \textbf{Software Engineering}: Development of automation scripts and validation systems
\item \textbf{Network-{protocols, services,  Administration of Systems and Services}}: Design of network architectures and IT infrastructure management
\item \textbf{Cybersecurity}: Implementation of ethical pentesting techniques and vulnerability analysis
\item \textbf{Project Management}: Planning, development, and delivery of a complete educational product
\end{itemize}

Obviously, there's a lot of other competencies involved, but those are the main ones that contribute to the successful development and implementation of the teaching package.

\subsection{Professional Context}

In the professional context, this project aligns with the growing demand for skilled cybersecurity professionals who can operate in complex virtual environments. The use of EVE-NG as a training platform reflects industry trends towards virtualization and cloud-based solutions. Furthermore, the focus on ethical pentesting techniques prepares students for real-world challenges, ensuring they are equipped with the necessary skills to succeed in the cybersecurity field. 

Despite on that, the main professional context is the educational one, as the project is aimed at improving the teaching and learning experience in cybersecurity education. 

\subsubsection{Technological Context}

EVE-NG is chosen as the base platform due to its ability to manage complex topologies, its compatibility with multiple virtualized operating systems, and its suitability for educational environments that require flexibility and reproducibility. The project leverages modern virtualization technologies and automation tools to create an efficient and effective learning environment for students.

Moreover, the use of widely adopted cybersecurity tools such as Kali Linux, Metasploit Framework, Nmap and digital forensic analysis software ensures that students gain hands-on experience with industry-standard technologies. This technological context not only enhances the learning experience but also prepares students for future professional roles in cybersecurity.

