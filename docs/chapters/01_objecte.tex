% 01_objecte.tex
% Document: TFG Avantprojecte - Object of the Project
% Idioma: Inglés

\chapter{Object of the Project}

\section{Problem Statement and Educational Gap}

Cybersecurity education faces a critical challenge: the disconnect between 
theoretical knowledge of IT systems and practical understanding of how 
security failures manifest in integrated environments. While students in 
the Computer Engineering degree program learn fundamental concepts---databases, 
network protocols, operating system internals, and software architecture---they 
rarely experience how misconfigurations and design flaws in these domains 
directly translate to security vulnerabilities.

Current training approaches amplify this gap. Commercial platforms such as 
HackTheBox and TryHackMe focus on isolated challenge-based scenarios that 
abstract away system complexity. Professional pentesting certifications 
(e.g., CEH, OSCP\cite{ceh_framework, oscp_exam}) expect hands-on experience in realistic environments, yet 
many academic programs struggle to provide such environments due to cost, 
complexity, or vendor lock-in.

Within the TecnoCampus Bachelor's Degree in Computer Engineering 
(Management and Information Systems track), the ``Introduction to Cybersecurity'' 
course currently lacks a structured, reusable, and automatable lab environment that:

\begin{itemize}
    \item Aligns pentesting scenarios with specific degree learning outcomes
    \item Enables students to recognize vulnerabilities within integrated systems 
          (not isolated exploits)
    \item Provides reproducible environments for consistent assessment
    \item Reduces instructor workload through automation and documentation
    \item Remains cost-effective and institution-independent (open-source, 
          locally maintained)
\end{itemize}

\section{Project Description and Scope}

This Final Degree Project develops a comprehensive, reusable teaching package 
of cybersecurity practical laboratories using EVE-NG (Emulated Virtual 
Environment - Next Generation) that directly addresses the identified gaps. 
The package is designed for the ``Introduction to Cybersecurity'' course 
within the Computer Engineering degree program.

\section{Deliverables}

The project consists of creating four structured virtualized labs covering 
primary pentesting domains:

\begin{enumerate}
    \item \textbf{Reconnaissance and Enumeration}: Network scanning, service 
          identification, and information gathering using industry-standard 
          tools (Nmap)
    
    \item \textbf{Web Application Vulnerabilities}: Practical exploitation 
          of OWASP Top 10\cite{owasp2021} categories within controlled, intentionally 
          vulnerable web applications
    
    \item \textbf{Network Traffic Analysis and Cryptography}: Packet capture 
          analysis using Wireshark, protocol dissection, and 
          encryption/decryption scenarios
    
    \item \textbf{Privilege Escalation}: Techniques for escalating privileges 
          on Unix/Linux and Windows systems, reflecting real-world 
          post-exploitation scenarios
\end{enumerate}

Each lab includes:

\begin{itemize}
    \item Complete network topologies with preconfigured virtual machines
    \item Automated deployment and reset scripts for efficient lab lifecycle management
    \item Technical documentation detailing network design, vulnerability 
          implementation, and remediation
    \item Student learning guides with structured objectives and self-assessment rubrics
    \item Assessment materials for instructors to evaluate lab completion 
          and skill acquisition
\end{itemize}

\section{Scope Boundaries}

\textbf{Included in this TFG:}

\begin{itemize}
    \item Design and implementation of 4 pentesting labs in EVE-NG
    \item Integration of real vulnerabilities and techniques aligned with 
          OWASP Top 10 and CIS Critical Controls\cite{cis_controls}
    \item Automation scripts for deployment, validation, and lab reset
    \item Complete technical documentation for instructors and future maintainers
    \item Student guides and learning objectives for each lab
    \item Assessment rubrics for evaluating student lab completion
\end{itemize}

\textbf{Explicitly NOT included (out of scope):}

\begin{itemize}
    \item Integration with third-party Learning Management Systems (LMS) or Moodle
    \item Development of proprietary lab management software or web interfaces
    \item Creation of automated grading systems beyond basic script validation
    \item Commercial licensing or distribution models
    \item Formal pedagogical evaluation studies 
          (e.g., pre/post learning assessments)
    \item Creation of labs beyond the four specified domains
\end{itemize}

\section{Why EVE-NG? Selection Rationale}

The selection of EVE-NG as the platform is justified through comparative 
analysis of alternatives and specific alignment with project requirements.

\section{Comparative Analysis}

\begin{enumerate}
    \item \textbf{HackTheBox}: 
    \begin{itemize}
        \item Strengths: Large collection of pre-built challenges, strong community
        \item Limitations: Pre-built scenarios; limited network topology customization; 
                external platform dependency; content aligned to commercial 
                pentesting certifications, not specific degree curricula
        \item Verdict: Unsuitable for curriculum-aligned, institution-controlled labs
    \end{itemize}
    
    \item \textbf{TryHackMe}: 
    \begin{itemize}
        \item Strengths: Gamified learning paths, beginner-friendly
        \item Limitations: Content managed externally; platform-dependent; 
                free tier is limited; scenarios often oversimplified for 
                professional preparation
        \item Verdict: Useful for introduction but insufficient for degree-level 
                practical integration
    \end{itemize}
    
    \item \textbf{SEED Labs}: 
    \begin{itemize}
        \item Strengths: Comprehensive, research-backed cybersecurity exercises; 
                Docker-based portability
        \item Limitations: Fixed scenarios; limited virtualization topology control; 
                minimal automation for lab reset
        \item Verdict: Good for specific security concepts but less suitable for 
                integrated network pentesting scenarios
    \end{itemize}
    
    \item \textbf{EVE-NG}: 
    \begin{itemize}
        \item Strengths: Full network topology customization; open-source; 
                locally deployable; supports multiple virtualized OS; 
                excellent automation capabilities; vendor-independent
        \item Limitations: Steeper learning curve; requires institutional IT infrastructure
        \item Verdict: Optimal for institution-controlled, curriculum-aligned, 
                reusable labs
    \end{itemize}
\end{enumerate}

\section{Key Advantages of EVE-NG}

\begin{enumerate}
    \item \textbf{Institutional Independence}: Open-source platform ensures 
          no vendor lock-in, long-term sustainability, and 
          institution-controlled customization
    
    \item \textbf{Customization and Control}: Full topology design allows labs 
          to reflect real infrastructure complexity (routing misconfigurations, 
          network segmentation failures, hybrid cloud scenarios)
    
    \item \textbf{Automation and Reproducibility}: Lab deployment, reset, and 
          validation can be fully automated, enabling consistent student 
          experiences and reducing instructor overhead
    
    \item \textbf{Curriculum Integration}: Labs can embed specific technologies 
          taught in parallel courses (OSPF routing, SQL databases, Active 
          Directory authentication), creating the nexus where theoretical 
          knowledge meets security contexts
    
    \item \textbf{Cost-Effectiveness}: Zero licensing costs; compatible with 
          open-source tools (Kali Linux, Metasploit Framework, Nmap); 
          appropriate for educational institutions with limited budgets
    
    \item \textbf{Scalability}: Lab files are portable and exportable, enabling 
          future deployment across multiple course sections, institutions, 
          or as a resource for the broader educational community
\end{enumerate}

\section{Alignment with Computer Engineering Degree}

This project creates an explicit connection between four core competencies 
taught throughout the degree program and their direct application to cybersecurity:

\begin{enumerate}
    \item \textbf{Programming and Software Development}: Labs demonstrate how 
          insecure coding practices (SQL injection, buffer overflows, improper 
          input validation) lead to exploitable vulnerabilities. Students 
          recognize that defensive programming is not an afterthought but a 
          core software engineering practice.
    
    \item \textbf{Software Engineering}: Network topologies and application 
          architectures in labs embody both secure and insecure design patterns. 
          Students analyze architectural decisions (monolithic vs. microservices, 
          authentication mechanisms, data separation) and observe their security 
          implications.
    
    \item \textbf{Systems and Information Engineering}: Labs integrate multiple 
          systems (databases, web servers, network services, operating systems) 
          in realistic configurations, reflecting how SI engineers must consider 
          threat landscapes when designing enterprise solutions.
    
    \item \textbf{Computer Architecture and Networks}: Privilege escalation labs 
          explore OS-level vulnerabilities, memory management flaws, and network 
          protocol weaknesses that emerge from architectural design decisions 
          taught in prerequisite courses.
\end{enumerate}

By demonstrating how theoretical knowledge directly maps to security vulnerabilities, 
this project transforms abstract concepts into concrete, experiential learning outcomes.

\subsection{Justification: Why This Project Matters}

\subsection{Pedagogical Importance}

\begin{itemize}
    \item \textbf{Theory-to-Practice Bridge}: Pentesting labs create the missing 
          link between theoretical coursework and professional-grade practical 
          experience
    
    \item \textbf{Active Learning}: Hands-on vulnerability discovery and 
          exploitation is more engaging and memorable than passive lectures 
          or textbook study
    
    \item \textbf{Professional Preparation}: Students gain experience with 
          industry-standard tools (Kali Linux, Metasploit, Wireshark) and 
          methodologies (OWASP, CIS Controls), directly preparing them for 
          professional roles
    
    \item \textbf{Assessment and Competency Verification}: Structured labs 
          with clear objectives and rubrics enable objective assessment of 
          student competency in cybersecurity fundamentals
\end{itemize}

\subsection{Institutional Importance}

\begin{itemize}
    \item \textbf{Course Enhancement}: The ``Introduction to Cybersecurity'' 
          course gains a modern, practical component that improves student 
          satisfaction and learning outcomes
    
    \item \textbf{Reduced Instructor Burden}: Automation scripts and comprehensive 
          documentation minimize preparation time, enabling instructors to focus 
          on learning support rather than lab administration
    
    \item \textbf{Reproducibility}: Consistent lab environments ensure fair 
          assessment across multiple course sections and cohorts
    
    \item \textbf{Scalability and Sustainability}: The reusable package can serve 
          multiple offerings of the course and potentially be adapted for related 
          courses in the degree program or distributed to other institutions
\end{itemize}

\subsection{Professional Context}

The cybersecurity industry demands professionals who understand both attack and 
defense across integrated systems. Practitioners must:

\begin{itemize}
    \item Recognize vulnerabilities within complex, heterogeneous environments 
          (not isolated systems)
    \item Apply defensive practices informed by knowledge of real attack methodologies
    \item Operate with professional ethics and legal compliance 
          (ethical hacking mindset)
    \item Use industry-standard tools and frameworks (OWASP, CIS, NIST)
\end{itemize}

This project prepares students to meet these professional expectations by 
providing realistic, integrated training environments.

\subsection{Technical and Technological Foundation}

\subsubsection{EVE-NG Platform}

EVE-NG provides the technical foundation for complex network topology simulation. 
Its capabilities include:

\begin{itemize}
    \item Support for multiple virtualization hypervisors (KVM, Docker, VMware)
    \item Native compatibility with Cisco IOS, Linux, Windows, and other 
          operating systems
    \item Advanced topology design tools enabling real-world network scenarios
    \item Scalability to support multi-lab environments for cohort-based learning
\end{itemize}

\subsubsection{Tooling Ecosystem}

Labs leverage industry-standard tools that students will encounter in professional roles:

\begin{itemize}
    \item \textbf{Kali Linux}: Primary offensive security platform with integrated 
          pentesting tools
    \item \textbf{Metasploit Framework}: Exploitation framework for demonstrating 
          real vulnerabilities
    \item \textbf{Nmap}: Network scanning and service enumeration
    \item \textbf{Wireshark}: Network protocol analysis and packet capture
    \item \textbf{OWASP ZAP}: Web application security scanner and proxy
    \item \textbf{Burp Suite Community}: Web application penetration testing
\end{itemize}

These tools are either free/open-source or have robust free tiers, ensuring 
long-term cost-effectiveness and institutional independence.

\subsection{Expected Outcomes and Success Criteria}

Upon completion, this TFG will deliver:

\begin{enumerate}
    \item \textbf{Four fully functional EVE-NG lab topologies} (.unl files) that 
          can be immediately deployed in the ``Introduction to Cybersecurity'' course
    
    \item \textbf{8-12 customized virtual machine images} with preconfigured 
          vulnerabilities, security tools, and network services
    
    \item \textbf{Automated deployment and validation scripts} that enable 
          one-command lab initialization and reset
    
    \item \textbf{Technical documentation} (50-100 pages) detailing network 
          architecture, vulnerability implementation, remediation strategies, 
          and maintenance procedures for future instructors
    
    \item \textbf{Student learning guides} for each lab with clear objectives, 
          step-by-step procedures, and self-assessment rubrics
    
    \item \textbf{Instructor assessment materials} including grading rubrics, 
          expected outcomes, and troubleshooting guides
    
    \item \textbf{A reproducible, reusable teaching package} that can serve 
          multiple course offerings, be adapted for related courses, and 
          potentially be shared with the broader educational community
\end{enumerate}

Success will be measured by:

\begin{itemize}
    \item Functionality: All labs deploy without errors and exhibit intended vulnerabilities
    \item Documentation completeness: Sufficient detail for instructors to maintain 
          and update labs independently
    \item Student usability: Learning guides are clear enough for autonomous lab 
          completion with minimal instructor intervention
    \item Institutional sustainability: The package can be maintained and updated 
          by TecnoCampus staff without external dependencies
    \item Transferability: Labs can be adapted to other courses or distributed 
          to other institutions with minimal modifications
\end{itemize}

\subsection{Conclusion}

This Final Degree Project addresses a concrete educational need by creating a 
modern, practical, institution-independent teaching resource that bridges 
theoretical coursework with professional-grade cybersecurity experience. By 
leveraging EVE-NG's flexibility and open-source nature, the project delivers a 
sustainable, scalable solution aligned with both the degree curriculum and 
professional industry standards.