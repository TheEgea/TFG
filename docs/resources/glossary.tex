% LaTeX Template for Bachelor's Thesis (TFG) Documentation
% Author: Eloi Egea Rada
% © 2025, Eloi Egea Rada
% This work is licensed under a Creative Commons Attribution-NonCommercial-ShareAlike 4.0 International License (CC BY-NC-SA 4.0).


\newglossaryentry{api_rest}{
name={API REST},
description={Application Programming Interface grounded in the \textit{Representational State Transfer} (REST) paradigm, leveraging the HTTP protocol to facilitate inter-system communication and enabling scalable, efficient execution of resource operations—querying, creation, modification, and deletion}
}

\newglossaryentry{pentesting}{
name={Pentesting},
description={Penetration testing; an authorised, controlled simulation of cyberattacks designed to identify security vulnerabilities across systems, networks, and applications. Fundamental component of this thesis}
}


\newglossaryentry{ethical_hacking}{
name={Ethical Hacking},
description={The authorised and legal practice of proactively exploiting security vulnerabilities with explicit permission, conducted within a framework of ethical and legal responsibility. Conceptual foundation of this educational endeavour}
}


\newglossaryentry{vulnerability}{
name={Vulnerability},
description={A flaw or weakness inherent in a system, application, or infrastructure that can be exploited by an attacker to gain unauthorised access, inflict damage, or compromise data integrity and confidentiality}
}


\newglossaryentry{exploit}{
name={Exploit},
description={A piece of code, technique, or procedure that strategically targets a specific vulnerability to achieve unauthorised access or execute arbitrary code on a target system}
}


\newglossaryentry{payload}{
name={Payload},
description={The component or data transmitted as part of an attack vector, encapsulating the malicious code or instructions the attacker intends to execute within the compromised system}
}



\newglossaryentry{reconnaissance}{
name={Reconnaissance},
description={The initial phase of a penetration test during which the attacker gathers intelligence about the target—encompassing systems, users, networks, and services—typically through passive techniques without direct system access. Lab 1 of this thesis}
}


\newglossaryentry{network_scanning}{
name={Network Scanning},
description={The process of discovering active hosts, open ports, and running services across a network using specialised tools such as Nmap. The foundational phase of offensive reconnaissance}
}


\newglossaryentry{enumeration}{
name={Enumeration},
description={A methodical process of systematically identifying and documenting network resources, active services, user accounts, and group memberships—typically following network scanning. Essential for pinpointing viable attack targets}
}


\newglossaryentry{owasp}{
name={OWASP},
description={\textit{Open Web Application Security Project}; an international community dedicated to advancing web application security. Publishes the industry-standard Top 10 catalogue of critical vulnerabilities and comprehensive testing methodologies}
}


\newglossaryentry{sql_injection}{
name={SQL Injection},
description={A critical vulnerability that permits an attacker to insert malicious SQL code into database queries, enabling unauthorised data access, modification, or deletion. Primary focus of Lab 2}
}


\newglossaryentry{cross_site_scripting}{
name={Cross-Site Scripting (XSS)},
description={A vulnerability enabling the injection of malicious JavaScript code into web pages accessed by other users, thereby compromising session tokens and exfiltrating sensitive personal data}
}

\newglossaryentry{wireshark}{
name={Wireshark},
description={An open-source network protocol analyser that captures and examines network traffic in real-time, essential for dissecting communications and uncovering security vulnerabilities. Primary tool for Lab 3}
}


\newglossaryentry{packet_capture}{
name={Packet Capture},
description={The process of intercepting and archiving data packets traversing a network, enabling the analysis of protocols, unencrypted communications, and suspicious data flows}
}


\newglossaryentry{network_topology}{
name={Network Topology},
description={The physical and logical arrangement of network components—including computers, servers, networking devices, and interconnections. Critical for designing realistic laboratory environments}
}



\newglossaryentry{privilege_escalation}{
name={Privilege Escalation},
description={A technique enabling a user with limited permissions to obtain elevated privileges—such as administrator or root access—thereby executing previously restricted operations. Central focus of Lab 4}
}


\newglossaryentry{root_access}{
name={Root Access},
description={Maximum-level access on Unix/Linux systems, equivalent to administrator privileges on Windows. A typical objective in attack chains and the culmination of exploitation phases}
}


\newglossaryentry{local_privilege_escalation}{
name={Local Privilege Escalation},
description={A privilege escalation technique executed locally on an already-compromised machine, exploiting vulnerabilities in the operating system kernel or misconfigured permissions}
}



\newglossaryentry{eve_ng}{
name={EVE-NG},
description={\textit{Emulated Virtual Environment - Next Generation}; a sophisticated simulation platform enabling the construction of intricate network topologies with diverse networking devices for educational and security testing purposes. Foundational platform for this thesis}
}


\newglossaryentry{kali}{
name={Kali Linux},
description={A purpose-built Linux distribution tailored for penetration testing and security auditing, encompassing hundreds of specialised tools for dictionary attacks, packet sniffing, vulnerability scanning, and related cybersecurity utilities. Primary operating system for attacker nodes in labs}
}


\newglossaryentry{virtual_machine}{
name={Virtual Machine},
description={A software-based emulation of a complete computer system executing within a physical host, enabling the concurrent execution of multiple operating systems and facilitating the creation of isolated, reproducible laboratory environments}
}


\newglossaryentry{dmz}{
name={DMZ (Demilitarized Zone)},
description={A neutral security segment segregating an organisation's internal network from the public Internet, hosting externally-accessible services whilst protecting critical internal resources from direct exposure}
}




\newglossaryentry{metasploit}{
name={Metasploit Framework},
description={An open-source framework facilitating the development, customisation, and execution of exploits against target systems. The industry standard for professional penetration testing engagements}
}


\newglossaryentry{nmap}{
name={Nmap},
description={An open-source network scanning utility that discovers active hosts, enumerates open ports, identifies running services, and fingerprints operating systems. Indispensable during reconnaissance phases}
}


\newglossaryentry{burp_suite}{
name={Burp Suite},
description={An integrated web application security testing platform enabling the interception, modification, and analysis of HTTP/HTTPS traffic. Community edition included in standard Kali Linux distributions}
}


\newglossaryentry{owasp_zap}{
name={OWASP ZAP},
description={An open-source utility designed for identifying vulnerabilities in web applications, offering both automated scanning capabilities and interactive testing workflows. A freely-available alternative to commercial solutions}
}


\newglossaryentry{cvss}{
name={CVSS (Common Vulnerability Scoring System)},
description={A standardised framework for quantifying vulnerability severity through a numerical score—ranging from 0 to 10—that reflects exploitability and potential impact on affected systems}
}


\newglossaryentry{vulnerability_assessment}{
name={Vulnerability Assessment},
description={A systematic process of identifying, quantifying, and prioritising security vulnerabilities across systems, networks, and applications. Typically precedes comprehensive penetration testing engagements}
}


\newglossaryentry{cis_controls}{
name={CIS Critical Controls},
description={An authoritative, consensus-based framework of 18 prioritised security controls representing best practices for defending against the most prevalent and effective attack vectors}
}


\newglossaryentry{nist_framework}{
name={NIST Cybersecurity Framework},
description={A standards-based framework from the National Institute of Standards and Technology for managing and mitigating cybersecurity risk, organised into five functional domains: Identify, Protect, Detect, Respond, and Recover}
}



\newglossaryentry{tfg}{
name={TFG (Bachelor's Thesis)},
description={A mandatory capstone project in university degree programmes, synthesising and demonstrating the comprehensive knowledge and competencies acquired throughout the academic programme}
}


\newglossaryentry{homelab}{
name={HomeLab},
description={A personal laboratory environment combining hardware and software infrastructure for educational purposes, experimentation, and professional development. Serves as the foundational infrastructure for this thesis}
}


\newglossaryentry{lab_topology}{
name={Lab Topology},
description={The architectural arrangement of virtual machines, network segments, and services within an educational laboratory environment, designed to faithfully simulate realistic penetration testing scenarios}
}
