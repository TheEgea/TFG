% LaTeX Template for Bachelor's Thesis (TFG) Documentation
% Author: Eloi Egea Rada
%
% © 2025, Eloi Egea Rada
% This work is licensed under a Creative Commons Attribution-NonCommercial-ShareAlike 4.0 International License (CC BY-NC-SA 4.0).


\newglossaryentry{api_rest}{
    name={API REST},
    description={Application Programming Interface grounded in the \textit{Representational State Transfer} (REST) paradigm, leveraging the HTTP protocol to facilitate inter-system communication and enabling scalable, efficient execution of resource operations—querying, creation, modification, and deletion}
}


% Additional entries relevant to the TFG on Penetration Testing
\newglossaryentry{pentesting}{
    name={Pentesting},
    description={Penetration testing; an authorised, controlled simulation of cyberattacks designed to identify security vulnerabilities across systems, networks, and applications before malicious actors can exploit them}
}


\newglossaryentry{eve_ng}{
    name={EVE-NG},
    description={\textit{Emulated Virtual Environment - Next Generation}; a sophisticated simulation platform enabling the construction of intricate network topologies with diverse networking devices, tailored for educational and security assessment purposes}
}


\newglossaryentry{kali}{
    name={Kali Linux},
    description={A purpose-built Linux distribution tailored for penetration testing and security auditing, encompassing hundreds of specialised tools for brute-force attacks, packet sniffing, vulnerability scanning, and related cybersecurity utilities}
}


\newglossaryentry{vulnerability}{
    name={Vulnerability},
    description={A flaw or weakness inherent in a system, application, or infrastructure that can be exploited by an attacker to gain unauthorised access, inflict damage, or compromise data integrity and confidentiality}
}


\newglossaryentry{exploit}{
    name={Exploit},
    description={A piece of code, technique, or procedure that strategically targets a specific vulnerability to achieve unauthorised access or execute arbitrary code on a target system}
}


\newglossaryentry{payload}{
    name={Payload},
    description={The component or data transmitted as part of an attack vector, encapsulating the malicious code or instructions the attacker intends to execute within the compromised system}
}


\newglossaryentry{reconnaissance}{
    name={Reconnaissance},
    description={The initial phase of a penetration test during which the attacker gathers intelligence about the target—encompassing systems, users, networks, and services—typically through passive techniques without direct system access}
}


\newglossaryentry{cvss}{
    name={CVSS},
    description={\textit{Common Vulnerability Scoring System}; a standardised framework for quantifying vulnerability severity through a numerical score that reflects exploitability and potential impact}
}


\newglossaryentry{owasp}{
    name={OWASP},
    description={\textit{Open Web Application Security Project}; an international community devoted to advancing web application security, renowned for publishing foundational resources including the \textit{Top 10} catalogue of critical vulnerabilities}
}


\newglossaryentry{nist_framework}{
    name={NIST Cybersecurity Framework},
    description={A comprehensive framework developed by the National Institute of Standards and Technology to systematically manage and mitigate cybersecurity risk across organisations}
}


\newglossaryentry{cis_controls}{
    name={CIS Critical Controls},
    description={A prioritised set of 18 defensive security practices that serve as foundational safeguards against the most prevalent and impactful cyber threats}
}


\newglossaryentry{network_topology}{
    name={Network Topology},
    description={The logical and physical arrangement of devices, nodes, and interconnections within a network infrastructure}
}


\newglossaryentry{enumeration}{
    name={Enumeration},
    description={The systematic and methodical process of identifying and cataloguing users, active services, and accessible resources within a target environment}
}


\newglossaryentry{privilege_escalation}{
    name={Privilege Escalation},
    description={A technique employed to obtain elevated permissions and access levels on a compromised system, often to achieve administrative or root-level control}
}


\newglossaryentry{metasploit}{
    name={Metasploit Framework},
    description={An open-source penetration testing platform facilitating the development, customisation, and execution of exploits against target systems}
}


\newglossaryentry{cryptography}{
    name={Cryptography},
    description={The mathematical science and practice of safeguarding information through encryption, hashing, and other cryptographic techniques}
}


\newglossaryentry{post_exploitation}{
    name={Post-Exploitation},
    description={The phase following initial system compromise, during which the attacker establishes persistence, extracts data, and escalates privileges to maintain long-term control}
}


\newglossaryentry{ethical_hacking}{
    name={Ethical Hacking},
    description={Authorised and responsibility-driven penetration testing conducted to proactively identify vulnerabilities before threat actors can weaponise them}
}
