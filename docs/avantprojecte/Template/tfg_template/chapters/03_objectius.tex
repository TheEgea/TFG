\section{Objectius i Abast}

\subsection{Objectiu Principal}

Desenvolupar, implementar i validar un paquet docent reutilitzable de laboratoris pràctics de ciberseguretat basat en EVE-NG, que permeti als estudiants de l'assignatura ``Introduction to Cybersecurity'' adquirir competències pràctiques en pentesting ètic mitjançant escenaris realistes i automatitzats.

\subsection{Objectius Específics (Mesurables amb KPIs)}

\subsubsection{OE1: Dissenyar i Implementar Laboratoris EVE-NG}

\textbf{Descripció}: Crear quatre laboratoris temàtics funcionals i interconnectats.

\textbf{KPIs Mesurables}:
\begin{itemize}
\item 4 topologies .unl completament funcionals (100\%)
\item Temps de desplegament $\leq$ 2 minuts per laboratori (95\% dels casos)
\item Taxa d'èxit en el boot de VMs $\geq$ 95\%
\item Compatibilitat amb EVE-NG Community i Professional editions
\end{itemize}

\textbf{Lliurables}:
\begin{itemize}
\item 4 fitxers .unl (topologies)
\item 8-12 imatges de màquines virtuals optimitzades
\item Documentació tècnica de configuració per laboratori
\end{itemize}

\subsubsection{OE2: Desenvolupar Scripts d'Automatització}

\textbf{Descripció}: Crear un sistema d'automatització complet per al cicle de vida dels laboratoris.

\textbf{KPIs Mesurables}:
\begin{itemize}
\item 12 scripts funcionals (3 per laboratori: deploy, reset, validate)
\item Temps d'execució del reset $\leq$ 30 segons per laboratori
\item Cobertura de validació automàtica $\geq$ 80\% dels components crítics
\item Zero errors en 10 execucions consecutives del cicle complet
\end{itemize}

\textbf{Lliurables}:
\begin{itemize}
\item Scripts de desplegament (Bash/Python)
\item Scripts de reset automatitzat
\item Scripts de validació i verificació d'estat
\end{itemize}

\subsubsection{OE3: Crear Material Docent Estructurat}

\textbf{Descripció}: Desenvolupar documentació completa per a estudiants i professorat.

\textbf{KPIs Mesurables}:
\begin{itemize}
\item 4 manuals d'usuari (1 per laboratori), mínim 15 pàgines cadascun
\item 4 rúbriques d'avaluació amb criteris quantificables
\item 1 guia de docent amb instruccions de desplegament
\item Temps de configuració inicial per docent $\leq$ 1 hora
\end{itemize}

\textbf{Lliurables}:
\begin{itemize}
\item Manual de l'alumne (anglès, format markdown/PDF)
\item Guia del professor amb instruccions detallades
\item Rúbriques d'avaluació estruturades
\item README tècnic amb requisits i instruccions
\end{itemize}

\subsubsection{OE4: Validar la Usabilitat i Eficàcia Educativa}

\textbf{Descripció}: Verificar que el paquet compleix els requisits educatius i tècnics.

\textbf{KPIs Mesurables}:
\begin{itemize}
\item Temps mitjà de resolució per laboratori: 90-120 minuts
\item Taxa de finalització exitosa $\geq$ 85\% (pilot amb 10 usuaris)
\item Puntuació de satisfacció usuaris $\geq$ 4/5
\item Zero incidències crítiques en entorn de producció
\end{itemize}

\textbf{Lliurables}:
\begin{itemize}
\item Informe de validació pilot
\item Mètriques d'usabilitat i rendiment
\item Recomanacions de millora implementades
\end{itemize}

\subsection{Definició del Client i Usuari Final}

\subsubsection{Client Principal}

\begin{itemize}
\item \textbf{Professor responsable}: Pere Vidiella i Catalan
\item \textbf{Assignatura}: Introduction to Cybersecurity (GEISI, Tecnocampus)
\item \textbf{Context}: Docència de grau en ciberseguretat pràctica
\end{itemize}

\subsubsection{Usuaris Finals Primaris}

\begin{itemize}
\item \textbf{Estudiants de GEISI}: 25-30 alumnes per edició
\item \textbf{Perfil}: Coneixements bàsics en xarxes i sistemes operatius
\item \textbf{Objectiu}: Aprendre pentesting ètic de forma pràctica i segura
\end{itemize}

\subsubsection{Usuaris Finals Secundaris}

\begin{itemize}
\item \textbf{Professorat de ciberseguretat}: Altres docents interessats en reutilitzar el material
\item \textbf{Estudiants de postgrau}: Possibles extensions del material per a nivells avançats
\end{itemize}

\subsection{Públic Potencial}

\subsubsection{Àmbit Intern (Tecnocampus)}

\begin{itemize}
\item Altres assignatures relacionades amb ciberseguretat
\item Projectes de recerca en seguretat informàtica
\item Formació contínua i certificacions professionals
\end{itemize}

\subsubsection{Àmbit Extern}

\begin{itemize}
\item Institucions educatives amb formació en ciberseguretat
\item Centres de formació professional especialitzats
\item Empreses amb programes de formació interna en seguretat
\end{itemize}

\subsection{KPIs i Indicadors Clau de Rendiment}

\subsubsection{Indicadors Tècnics}

\begin{table}[h!]
\centering
\begin{tabular}{|l|l|l|}
\hline
\textbf{Mètrica} & \textbf{Objectiu} & \textbf{Mètode de Mesura} \\
\hline
Temps desplegament laboratori & $\leq$ 2 min & Cronometratge automatitzat \\
\hline
Taxa d'èxit boot VMs & $\geq$ 95\% & Logs de sistema + scripts validació \\
\hline
Temps reset complet & $\leq$ 30 seg & Scripts amb timestamps \\
\hline
Reproducibilitat escenaris & 100\% & Tests automatitzats \\
\hline
\end{tabular}
\end{table}

\subsubsection{Indicadors Educatius}

\begin{table}[h!]
\centering
\begin{tabular}{|l|l|l|}
\hline
\textbf{Mètrica} & \textbf{Objectiu} & \textbf{Mètode de Mesura} \\
\hline
Temps resolució laboratori & 90-120 min & Tracking temporal en pilot \\
\hline
Taxa de finalització exitosa & $\geq$ 85\% & Seguiment progressió usuaris pilot \\
\hline
Satisfacció usuaris & $\geq$ 4/5 & Enquesta post-utilització \\
\hline
Comprensió conceptes clau & $\geq$ 80\% correcte & Qüestionari d'avaluació \\
\hline
\end{tabular}
\end{table}

\subsubsection{Indicadors de Qualitat}

\begin{table}[h!]
\centering
\begin{tabular}{|l|l|l|}
\hline
\textbf{Mètrica} & \textbf{Objectiu} & \textbf{Mètode de Mesura} \\
\hline
Cobertura documentació & 100\% funcionalitats & Checklist de verificació \\
\hline
Errors crítics & 0 & Testing exhaustiu \\
\hline
Compatibilitat versions EVE-NG & 100\% & Tests en múltiples entorns \\
\hline
Usabilitat interfície & $\geq$ 4/5 & Avaluació heurística UX \\
\hline
\end{tabular}
\end{table}

\noindent \textbf{Nota}: Tots els KPIs seran mesurats durant la fase de validació pilot (abril-maig 2026) i documentats en l'informe final del TFG.