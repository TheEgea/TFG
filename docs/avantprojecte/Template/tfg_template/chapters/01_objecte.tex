


% 01_objecte.tex
% Document: TFG Template
\section{Objecte del TFG}

\subsection{Descripció del Projecte}

Aquest Treball Final de Grau té com a finalitat el desenvolupament d'un paquet docent reutilitzable de laboratoris pràctics de ciberseguretat utilitzant la plataforma EVE-NG (Emulated Virtual Environment - Next Generation) per a l'assignatura ``Introduction to Cybersecurity'' del Grau en Enginyeria Informàtica de Gestió i Sistemes d'Informació.

El projecte consisteix en la creació d'un conjunt estructurat de quatre laboratoris virtualitzats que cobreixen les principals àrees del pentesting ètic: reconeixement i enumeració, vulnerabilitats d'aplicacions web, anàlisi de tràfic i criptografia, i escalada de privilegis. Cada laboratori inclourà topologies de xarxa completes amb màquines virtuals preconfigurades, scripts d'automatització per al desplegament i reset d'escenaris, i documentació tècnica exhaustiva.

\subsection{Argumentació i Justificació}

\subsubsection{Context Educatiu i Necessitat Identificada}

La formació en ciberseguretat requereix d'entorns pràctics segurs on els estudiants puguin experimentar amb tècniques de pentesting sense riscos per a sistemes reals. Actualment, l'assignatura ``Introduction to Cybersecurity'' del Tecnocampus disposa d'infraestructura virtualitzada desenvolupada en projectes anteriors, però manca d'un paquet docent estructurat, automatitzat i reutilitzable que faciliti tant la docència com l'aprenentatge autònom.

\subsubsection{Valor Afegit i Innovació}

Aquest TFG aporta valor en múltiples dimensions:

\begin{enumerate}
\item \textbf{Reutilització i Escalabilitat}: El paquet docent està dissenyat per ser utilitzat en múltiples edicions de l'assignatura, reduint la càrrega de preparació del professorat.
\item \textbf{Automatització del Cicle de Vida}: Els scripts de desplegament, reset i validació permeten una gestió eficient dels laboratoris, minimitzant el temps dedicat a tasques administratives.
\item \textbf{Aprenentatge Autoguiat}: La documentació estructurada i les rúbriques d'avaluació faciliten l'aprenentatge autònom dels estudiants.
\item \textbf{Estàndards Professionals}: Els escenaris reprodueixen vulnerabilitats i tècniques reals utilitzades en l'àmbit professional de la ciberseguretat.
\end{enumerate}

\subsubsection{Impacte Esperat}

La implementació d'aquest paquet docent permetrà:

\begin{itemize}
\item Millorar l'experiència d'aprenentatge dels estudiants mitjançant pràctiques realistes
\item Reduir el temps de preparació i gestió dels laboratoris per part del professorat
\item Facilitar la reproducibilitat i consistència en l'avaluació dels estudiants
\item Establir una base sòlida per a futurs desenvolupaments en formació de ciberseguretat
\end{itemize}

\subsection{Marc de Referència}

\subsubsection{Context Acadèmic}

Aquest projecte s'emmarca dins l'àmbit de l'enginyeria informàtica aplicada a l'educació, combinant competències tècniques de:

\begin{itemize}
\item \textbf{Enginyeria del Software}: Desenvolupament d'scripts d'automatització i sistemes de validació
\item \textbf{Sistemes d'Informació}: Disseny d'arquitectures de xarxa i gestió d'infrastructura IT
\item \textbf{Ciberseguretat}: Implementació de tècniques de pentesting ètic i anàlisi de vulnerabilitats
\item \textbf{Gestió de Projectes}: Planificació, desenvolupament i lliurament d'un producte educatiu complet
\end{itemize}
administracion de sistemas y servicios en ingles es: 

\subsubsection{Context Professional}

El projecte respon a la necessitat creixent de professionals qualificats en ciberseguretat, proporcionant als estudiants experiència pràctica amb eines i tècniques estandarditzades en la industria, com ara Kali Linux, Metasploit Framework, Nmap, Burp Suite i anàlisi forense digital.

\subsubsection{Context Tecnològic}

S'utilitza EVE-NG com a plataforma base degut a la seva capacitat per gestionar topologies complexes, la seva compatibilitat amb múltiples sistemes operatius virtualitzats, i la seva idoneïtat per a entorns educatius que requereixen flexibilitat i reproductibilitat.

% 01_objecte.tex
% Document: TFG Template 
% Idioma: Inglés
\section{Objective of the TFG}
\subsection{Project Description}

The objective of this Final Degree Project is to develop a reusable teaching package of cybersecurity practical labs using the EVE-NG (Emulated Virtual Environment - Next Generation) platform for the course "Introduction to Cybersecurity" of the Bachelor's Degree in Computer Engineering in Management and Information Systems.

The project consists of creating a structured set of four virtualized labs covering the main areas of ethical pentesting: reconnaissance and enumeration, web application vulnerabilities, traffic analysis and cryptography, and privilege escalation. Each lab will include complete network topologies with preconfigured virtual machines, automation scripts for scenario deployment and reset, and comprehensive technical documentation.

\subsection{Rationale and Justification}

\subsection{Educational Context and Identified Needs}

Cybersecurity training requires safe practical environments where students can experiment with pentesting techniques without risks to real systems. Currently, the "Introduction to Cybersecurity" course at Tecnocampus has virtualized infrastructure developed in previous projects, but lacks a structured, automated, and reusable teaching package that facilitates both teaching and autonomous learning.

\subsubsection{Added Value and Innovation}
This TFG adds value in multiple dimensions:
\begin{enumerate}
\item \textbf{Reusability and Scalability}: The teaching package is designed to be used in multiple editions of the course, reducing the preparation load for the teaching staff.
\item \textbf{Automation of the Life Cycle}: Deployment, reset, and validation scripts allow efficient lab management, minimizing time spent on administrative tasks.
\item \textbf{Self-Guided Learning}: Structured documentation and assessment rubrics facilitate autonomous learning for students.
\item \textbf{Professional Standards}: The scenarios reproduce real vulnerabilities and techniques used in the professional field of cybersecurity.
\end{enumerate}

\subsection{Expected Outcomes}

The implementation of this teaching package will enable:

\begin{itemize}
\item Improved student learning experiences through realistic practices
\item Reduced preparation and management time for teachers
\item Enhanced reproducibility and consistency in student assessment
\item Establishment of a solid foundation for future developments in cybersecurity training
\end{itemize}

\subsection{Reference Framework}

\subsubsection{Academic Context}

This project is framed within the field of computer engineering applied to education, combining technical competencies in:

\begin{itemize}
\item \textbf{Software Engineering}: Development of automation scripts and validation systems
\item \textbf{Network-{protocols, services, }}: Design of network architectures and IT infrastructure management
\item \textbf{Cybersecurity}: Implementation of ethical pentesting techniques and vulnerability analysis
\item \textbf{Project Management}: Planning, development, and delivery of a complete educational product
\end{itemize}



























