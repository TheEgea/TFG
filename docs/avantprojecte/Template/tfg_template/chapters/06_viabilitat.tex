\chapter{Feasibility Study}

\section{Initial Planning}

\subsection{Task Definition}

The project has been structured in 37 tasks distributed across three main phases: Preparation (Phase 0), Avantprojecte (Phase 1), and Implementation and Final Delivery (Phases 2 and 3). Each task has been carefully defined with the following attributes:

\begin{itemize}
    \item \textbf{Task identifier}: Unique numeric code (e.g., 0.1, 1.0, 2.1, 3.4)
    \item \textbf{Task name}: Descriptive title of the activity
    \item \textbf{Category}: Classification (Preparation, Analysis, Documentation, Development, Testing, Implementation, Milestones)
    \item \textbf{Estimated hours}: Time allocation based on complexity and dependencies
    \item \textbf{Start and end dates}: Temporal boundaries for execution
    \item \textbf{Completion percentage}: Current progress indicator
    \item \textbf{Critical path flag}: Identification of tasks that directly impact delivery deadlines
\end{itemize}

\subsubsection{Task Categories}

The 37 tasks have been distributed across 7 categories:

\begin{table}[H]
    \centering
    \begin{tabular}{|l|r|r|}
    \hline
    \textbf{Category} & \textbf{Number of Tasks} & \textbf{Total Hours} \\
    \hline
    Preparation & 4 & 105h \\
    Analysis & 5 & 49h \\
    Documentation & 6 & 95h \\
    Development & 8 & 178h \\
    Testing & 5 & 123h \\
    Implementation & 6 & 220h \\
    Milestones & 3 & 95h \\
    \hline
    \textbf{TOTAL} & \textbf{37} & \textbf{820h} \\
    \hline
    \end{tabular}
    \caption{Distribution of tasks by category}
\end{table}

\subsubsection{Phase Breakdown}

\textbf{Phase 0: Preparation (July - November 2025)}

\begin{itemize}
    \item Task 0.1: HomeLab Initialization Setup (64h) - \textbf{100\% completed}
    \item Task 0.2: VSCode Setup for thesis development (18h) - \textbf{100\% completed}
    \item Task 0.3: Kali Linux Setup and Basic Tools (15h) - \textbf{100\% completed}
    \item Task 0.4: Home Lab As-Built Documentation (8h) - \textbf{100\% completed}
\end{itemize}

This phase established the technical infrastructure necessary for project development, including virtualization environment (EVE-NG), development tools (VSCode, LaTeX), and pentesting platform (Kali Linux).

\textbf{Phase 1: Avantprojecte (October 2025 - January 2026)}

\begin{itemize}
    \item Task 1.0: Thesis Development (22h) - \textbf{100\% completed}
    \item Task 1.1: SMART Objectives Definition (4h) - \textbf{100\% completed}
    \item Task 1.2: State of the Art: Frameworks and Tools (15h) - \textbf{75\% in progress}
    \item Task 1.3: Functional and Non-Functional Requirements (10h) - \textbf{80\% in progress}
    \item Task 1.4: Pentesting Methodology and Validation (8h) - \textbf{0\% pending - CRITICAL}
    \item Task 1.5: Design Decisions and Justification (5h) - \textbf{0\% pending}
    \item Task 1.6: Gantt Chart Development (20h) - \textbf{100\% completed}
    \item Task 1.7: Risk Analysis and Mitigation Plan (15h) - \textbf{0\% pending}
    \item Task 1.8: Resource Inventory (5h) - \textbf{0\% pending}
    \item Task 1.9: Hackathon CTF Participation (25h) - \textbf{40\% in progress}
    \item \textbf{Milestone 1.10: Preliminary Project (40h)} - \textbf{50\% in progress - Delivery: January 16, 2026}
\end{itemize}

This phase focuses on conceptual design, requirements analysis, and detailed planning. Critical tasks for the January 16 deadline include completing the pentesting methodology, risk analysis, and resource inventory.

\textbf{Phase 2: Interim Report (January - March 2026)}

\begin{itemize}
    \item Task 2.1: Lab 1 Research - Reconnaissance (14h) - \textbf{CRITICAL}
    \item Task 2.2: Lab 2 Research - Web Vulnerabilities (14h) - \textbf{CRITICAL}
    \item Task 2.3: Lab 3 Research - Privilege Escalation (14h) - \textbf{CRITICAL}
    \item Task 2.4: Technical Write-ups Labs 1 \& 2 (13h)
    \item Task 2.5: Topology Adjustments (30h)
    \item Task 2.6: Pre-validation with Pilot Users (20h) - \textbf{CRITICAL}
    \item Task 2.6.1: Pre-validation Revision (6h) - \textbf{CRITICAL}
    \item Task 2.7: Analyze and Write Hackathon Results (17h)
    \item Task 2.8: Interim Report Formalization (40h)
    \item Task 2.9: Set Up Servers, VMs and EVE-NG (20h)
    \item Task 2.10: Backup and Documentation of EVE-NG (10h)
    \item \textbf{Milestone 2.11: Intermediate Thesis (50h)} - \textbf{Delivery: March 20, 2026}
\end{itemize}

This phase implements the first three laboratories and validates them with pilot users. Total estimated hours: 212h.

\textbf{Phase 3: Final Delivery (March - May 2026)}

\begin{itemize}
    \item Task 3.1: Lab 4 Research - Post-Exploitation (16h) - \textbf{CRITICAL}
    \item Task 3.2: Lab 3 Validation with Users (12h)
    \item Task 3.3: Technical Write-ups Labs 3 \& 4 (19h)
    \item Task 3.4: Penetration Testing Phase 1 (Labs 1-2-3) (30h) - \textbf{CRITICAL}
    \item Task 3.5: Penetration Testing Phase 2 (All Labs) (25h) - \textbf{CRITICAL}
    \item Task 3.6: Python Automation Scripts (30h)
    \item Task 3.7: Final Validation with Pilot Users (18h)
    \item Task 3.8: Final Review and Project Polishing (25h)
    \item Task 3.9: TFG Report Writing I (15h) - \textbf{CRITICAL}
    \item Task 3.10: TFG Report Writing II (40h) - \textbf{CRITICAL}
    \item Task 3.11: TFG Report Writing III (35h) - \textbf{CRITICAL}
    \item Task 3.12: Final Revision (15h)
    \item Task 3.13: Final Report Preparation (20h)
    \item \textbf{Milestone 3.14: Final Report (5h)} - \textbf{Delivery: May 27, 2026}
\end{itemize}

This phase completes the fourth laboratory, performs exhaustive testing, develops automation scripts, and writes the final report. Total estimated hours: 315h.

\subsection{Gantt Chart}

The complete project planning has been managed using an interactive web-based Gantt chart application developed specifically for this TFG. The tool provides:

\begin{itemize}
    \item Visualization of all 37 tasks across the project timeline
    \item Identification of 13 critical path tasks
    \item Progress tracking through completion percentages
    \item Task categorization into 7 groups
    \item CSV import/export functionality
    \item High-resolution PNG export for documentation
    \item Version history tracking
\end{itemize}
\newpage
\begin{figure}[H]
    \centering
    \includegraphics[width=\textwidth]{images/gantt_planning.png}
    \caption{Complete Gantt chart of TFG project planning (exported from interactive tool)}
    \label{fig:gantt_complete}
\end{figure}


\subsection{Critical Paths}

Critical path tasks are those that directly impact the delivery deadlines and cannot be delayed without affecting subsequent milestones. The project has identified \textbf{13 critical tasks}:

\begin{table}[H]
    \centering
    \small
    \begin{tabularx}{\textwidth}{|l|X|c|c|}
    \hline
    \textbf{ID} & \textbf{Critical Task} & \textbf{Hours} & \textbf{Deadline} \\
    \hline
    1.4 & Pentesting Methodology and Validation & 8h & Nov 26, 2025 \\
    1.10 & Milestone: Preliminary Project & 40h & Jan 16, 2026 \\
    2.1 & Lab 1 Research & 14h & Jan 26, 2026 \\
    2.2 & Lab 2 Research & 14h & Jan 31, 2026 \\
    2.3 & Lab 3 Research & 14h & Feb 7, 2026 \\
    2.6 & Pre-validation with Pilot Users & 20h & Feb 23, 2026 \\
    2.6.1 & Pre-validation Revision & 6h & Mar 1, 2026 \\
    3.1 & Lab 4 Research & 16h & Mar 23, 2026 \\
    3.4 & Penetration Testing Phase 1 & 30h & Apr 16, 2026 \\
    3.5 & Penetration Testing Phase 2 & 25h & Apr 30, 2026 \\
    3.9 & TFG Report Writing I & 15h & May 1, 2026 \\
    3.10 & TFG Report Writing II & 40h & May 20, 2026 \\
    3.11 & TFG Report Writing III & 35h & May 25, 2026 \\
    \hline
    \multicolumn{2}{|l|}{\textbf{TOTAL CRITICAL PATH}} & \textbf{277h} & \\
    \hline
    \end{tabularx}
    \caption{Critical path tasks requiring strict adherence to deadlines}
\end{table}

Any delay in these tasks will cascade to subsequent activities and potentially jeopardize milestone delivery dates.
 Special monitoring and contingency planning are allocated to these tasks.

\section{Budget}

\subsection{Cost of Human Resources}

The total estimated effort for the TFG is \textbf{820 hours}, corresponding to the 20 ECTS credits
 assigned to the subject (25 hours per ECTS). The distribution is as follows:

\begin{table}[H]
    \centering
    \begin{tabular}{|l|r|r|r|}
    \hline
    \textbf{Phase} & \textbf{Hours} & \textbf{Hourly Rate} & \textbf{Total Cost} \\
    \hline
    Phase 0: Preparation & 105h & 15 \euro/h & 1,575 \euro \\
    Phase 1: Avantprojecte & 188h & 15 \euro/h & 2,820 \euro \\
    Phase 2: Interim Report & 212h & 15 \euro/h & 3,180 \euro \\
    Phase 3: Final Delivery & 315h & 15 \euro/h & 4,725 \euro \\
    \hline
    \textbf{TOTAL} & \textbf{820h} & & \textbf{12,300 \euro} \\
    \hline
    \end{tabular}
    \caption{Human resources cost estimation based on junior engineer hourly rate}
\end{table}

The hourly rate of 15 \euro/hour corresponds to a junior cybersecurity engineer or IT student performing
 practical work, based on market rates in Spain for internships and entry-level positions in 2025-2026.
 Despite this, the cost remains theoretical, as the TFG is not a commercial
 project but an academic exercise. Also, considering the educational context,
 this cost is not billed to any entity and less being 15 €/hour.

\subsection{Cost of Hardware and Software Resources}

\subsubsection{Hardware Infrastructure}

\begin{table}[H]
    \centering
    \begin{tabular}{|l|r|r|r|}
    \hline
    \textbf{Resource} & \textbf{Quantity} & \textbf{Unit Cost} & \textbf{Total} \\
    \hline
    HomeLab Server (CPU 28-core, 32GB RAM, +32TB SSD) & 1 & 800 \euro & 800 \euro \\
    Development Workstation (Laptop + Desktop Computer) & 1 & 1200 \euro & 1200 \euro \\
    Network Equipment (Router, Switch) & 1 set & 150 \euro & 150 \euro \\
    External Storage (Backup) & 1 & 100 \euro & 100 \euro \\
    \hline
    \textbf{SUBTOTAL Hardware} & & & \textbf{2,250 \euro} \\
    \hline
    \end{tabular}
    \caption{Hardware infrastructure costs}
\end{table}
\textbf{Note}: The HomeLab server is a one-time investment that will be reused for future projects and courses,
 amortizing its cost over multiple years. The development workstation and network equipment are also reusable assets.
 Although that the use of the HomeLab server, a laptop and a desktop computer is already assumed on Phase 0,
 and its cost is not directly billed to the project, it's included here for completeness.
\subsubsection{Software and Licenses}

\begin{table}[H]
    \centering
    \begin{tabular}{|l|r|r|r|}
    \hline
    \textbf{Resource} & \textbf{License Type} & \textbf{Unit Cost} & \textbf{Total} \\
    \hline
    EVE-NG Community Edition & Free/Open Source & 0 \euro & 0 \euro \\
    Kali Linux & Free/Open Source & 0 \euro & 0 \euro \\
    VMware Workstation Pro (Educational) & Educational License & 0 \euro & 0 \euro \\
    Metasploitable VMs & Free/Open Source & 0 \euro & 0 \euro \\
    DVWA (Damn Vulnerable Web App) & Free/Open Source & 0 \euro & 0 \euro \\
    Visual Studio Code & Free/Open Source & 0 \euro & 0 \euro \\
    LaTeX (TeX Live) & Free/Open Source & 0 \euro & 0 \euro \\
    Git + GitHub (Student Pack) & Free/Student & 0 \euro & 0 \euro \\
    Burp Suite Community & Free & 0 \euro & 0 \euro \\
    Wireshark & Free/Open Source & 0 \euro & 0 \euro \\
    \hline
    \textbf{SUBTOTAL Software} & & & \textbf{0 \euro} \\
    \hline
    \end{tabular}
    \caption{Software and license costs (all free/open-source for educational use)}
\end{table}

A significant advantage of this project is the reliance on free and open-source tools,
 which reduces costs to zero for software licensing while maintaining professional-grade capabilities.

\subsection{Other Costs}
%there's no cost on printing and binding, as it's all digital submission
\begin{table}[H]
    \centering
    \begin{tabular}{|l|r|}
    \hline
    \textbf{Concept} & \textbf{Cost} \\
    \hline
    Electricity consumption (820h × 0.15 kWh × 0.20 \euro/kWh) & 25 \euro \\
    Internet connectivity (9 months × 40 \euro/month) & 360 \euro \\
    Documentation printing and binding (3 copies) & 60 \euro \\
    Contingency fund (5\% of total) & 635 \euro \\
    \hline
    \textbf{TOTAL Other Costs} & \textbf{1,080 \euro} \\
    \hline
    \end{tabular}
    \caption{Additional operational costs}
\end{table}

\subsection{Total Budget Summary}

\begin{table}[H]
    \centering
    \begin{tabular}{|l|r|}
    \hline
    \textbf{Category} & \textbf{Total Cost} \\
    \hline
    Human Resources & 12,300 \euro \\
    Hardware Resources & 2,250 \euro \\
    Software Resources & 0 \euro \\
    Other Costs & 1,080 \euro \\
    \hline
    \textbf{TOTAL PROJECT BUDGET} & \textbf{15,630 \euro} \\
    \hline
    \end{tabular}
    \caption{Complete project budget breakdown}
\end{table}

\section{Feasibility Analysis}

\subsection{Technical Feasibility}

\subsubsection{Infrastructure Availability}

The project demonstrates \textbf{high technical feasibility} due to the following factors:

\begin{itemize}
    \item \textbf{Completed Infrastructure Setup}: Phase 0 (Preparation) has been 100\% completed, with HomeLab server, EVE-NG platform, Kali Linux, and development environment fully operational and documented.
    
    \item \textbf{Open-Source Ecosystem}: All software components (EVE-NG, Kali Linux, vulnerable VMs, pentesting tools) are mature, well-documented, and widely adopted in professional cybersecurity training.
    
    \item \textbf{Proven Technology Stack}: The selected technologies (VMware/EVE-NG for virtualization, Python/Bash for scripting, LaTeX for documentation) are industry-standard and have established best practices.
    
    \item \textbf{Academic Support}: Access to institutional resources at Tecnocampus, including academic licenses for software (VMware Workstation Pro), library resources for research, and tutor expertise in cybersecurity and network administration.
\end{itemize}

\subsubsection{Technical Competencies}

The project requires competencies in multiple areas:

\begin{table}[H]
    \centering
    \begin{tabularx}{\textwidth}{|l|X|c|}
    \hline
    \textbf{Area} & \textbf{Required Competency} & \textbf{Level} \\
    \hline
    Network Administration & Configuration of virtual networks, VLANs, routing, firewall rules & Intermediate \\
    Operating Systems & Linux administration, Windows Server configuration & Intermediate \\
    Virtualization & EVE-NG topology design, VM management, resource optimization & Advanced \\
    Pentesting & Reconnaissance, exploitation, privilege escalation, web vulnerabilities & Intermediate \\
    Scripting & Python and Bash for automation and validation & Intermediate \\
    Documentation & LaTeX, technical writing, academic formatting & Intermediate \\
    \hline
    \end{tabularx}
    \caption{Required technical competencies and current proficiency level}
\end{table}

The student has demonstrated competence in these areas through completed courses
 (Network Administration, Operating Systems, Software Engineering) and the successful completion of
  Phase 0 infrastructure setup.

\subsubsection{Risk Mitigation}

Technical risks have been identified and mitigation strategies established:

\begin{itemize}
    \item \textbf{EVE-NG Performance Issues}: Risk mitigated through hardware dimensioning (32GB RAM, 8-core CPU), resource optimization techniques, and documented fallback to VirtualBox if needed.
    
    \item \textbf{VM Incompatibility}: Risk mitigated through early testing of all VM images (Metasploitable, DVWA, Kali Linux), use of standardized formats (QCOW2, VMDK), and maintenance of alternative image repository.
    
    \item \textbf{Knowledge Gaps}: Risk mitigated through structured research phase (Task 1.2: State of the Art, 15h), continuous learning during implementation, and access to tutor expertise.
\end{itemize}

\textbf{Conclusion}: The project is \textbf{technically viable} with a well-prepared infrastructure, appropriate technology stack, and adequate competency level. The main technical challenges are manageable with proper planning and contingency measures.

\subsection{Economic Feasibility}

\subsubsection{Cost-Benefit Analysis}

The project presents \textbf{excellent economic feasibility} for the following reasons:

\begin{itemize}
    \item \textbf{Zero Software Licensing Costs}: The exclusive use of free and open-source software eliminates recurring licensing expenses, making the solution sustainable for long-term educational use.
    
    \item \textbf{Low Infrastructure Investment}: The hardware infrastructure (2,250 \euro) is a one-time investment that can be reused across multiple academic years and courses, amortizing costs over 3-5 years.
    
    \item \textbf{Reusability and Scalability}: The teaching package is designed to be reused in multiple course editions (Introduction to Cybersecurity) with minimal maintenance costs, generating value beyond the initial development.
    
    \item \textbf{Institutional Value}: The deliverable provides long-term value to Tecnocampus by establishing a reproducible, automated laboratory infrastructure that reduces preparation time for faculty and enhances learning outcomes for students.
\end{itemize}

\subsubsection{Cost Comparison with Alternatives}

\begin{table}[H]
    \centering
    \small
    \begin{tabularx}{\textwidth}{|l|X|r|}
    \hline
    \textbf{Alternative} & \textbf{Description} & \textbf{Annual Cost} \\
    \hline
    This Project & Custom EVE-NG labs with automation, tailored to curriculum & 0 \euro/year* \\
    HackTheBox Academy & Commercial platform with pre-built labs and guided exercises & 1,200 \euro/year \\
    TryHackMe Education & Gamified learning platform with structured learning paths & 800 \euro/year \\
    Cybrary for Teams & Video-based training with labs and certification prep & 1,500 \euro/year \\
    Custom Enterprise Labs & Outsourced lab development by specialized company & 8,000-15,000 \euro \\
    \hline
    \end{tabularx}
    \caption{Cost comparison with commercial alternatives (*excludes initial development)}
\end{table}

While commercial platforms offer convenience, they lack customization to specific curriculum needs, do not integrate with institutional learning management systems, and incur recurring annual costs. This project provides a tailored, cost-effective, and sustainable solution.

\textbf{Conclusion}: The project demonstrates \textbf{high economic feasibility} with minimal ongoing costs, significant institutional value, and favorable comparison to commercial alternatives.

\subsection{Environmental Feasibility}

\subsubsection{Energy Consumption Analysis}

The environmental impact of this project is primarily related to energy consumption during development and operation:

\begin{itemize}
    \item \textbf{Development Phase (820 hours)}: 
    \begin{itemize}
        \item HomeLab server power consumption: 150W average
        \item Development workstation: 65W average
        \item Network equipment: 15W
        \item Total power: 230W × 820h = 188.6 kWh
        \item CO₂ emissions (Spain avg. 0.25 kg CO₂/kWh): 47.15 kg CO₂
    \end{itemize}
    
    \item \textbf{Operational Phase (per academic year)}:
    \begin{itemize}
        \item Estimated usage: 4 laboratory sessions × 30 students × 3 hours = 360 hours/year
        \item Power consumption: 230W × 360h = 82.8 kWh/year
        \item CO₂ emissions: 20.7 kg CO₂/year
    \end{itemize}
\end{itemize}

\subsubsection{Comparison with Traditional Alternatives}

\begin{table}[H]
    \centering
    \begin{tabularx}{\textwidth}{|l|X|r|}
    \hline
    \textbf{Scenario} & \textbf{Description} & \textbf{Annual CO₂} \\
    \hline
    This Project & Virtualized labs on shared HomeLab infrastructure & 20.7 kg \\
    Individual VMs & Each student runs labs on personal computer (30 × 200W × 12h) & 180 kg \\
    Cloud-Based Labs & AWS/Azure hosted environments (datacenter PUE 1.5) & 95 kg \\
    Physical Network Lab & Dedicated hardware devices (switches, routers, servers) & 450 kg \\
    \hline
    \end{tabularx}
    \caption{CO₂ emissions comparison across different laboratory approaches}
\end{table}

\subsubsection{Sustainability Measures}

The project incorporates several sustainability practices:

\begin{itemize}
    \item \textbf{Resource Optimization}: VMs are configured with minimal resource allocation, suspended when idle, and share infrastructure efficiently through virtualization.
    
    \item \textbf{Digital Documentation}: All documentation is maintained in digital format (LaTeX source, PDF outputs), avoiding paper waste except for required printed thesis copies (3 units).
    
    \item \textbf{Hardware Longevity}: The HomeLab server is designed for longevity (3-5 years) with modular components that can be upgraded rather than replaced entirely.
    
    \item \textbf{Open-Source Philosophy}: Reliance on open-source software reduces electronic waste by extending the lifespan of existing hardware that might be considered obsolete for commercial software requirements.
\end{itemize}

\textbf{Conclusion}: The project presents \textbf{good environmental feasibility} with significantly lower carbon footprint compared to physical lab alternatives or individual VM deployments. The virtualized approach maximizes resource efficiency and minimizes environmental impact.

\subsection{Legal Aspects}

\subsubsection{Intellectual Property and Licensing}

The project complies with intellectual property regulations:

\begin{itemize}
    \item \textbf{Open-Source Software}: All utilized software (EVE-NG, Kali Linux, Metasploitable, DVWA) is licensed under permissive open-source licenses (GPL, MIT, Apache 2.0) that explicitly allow educational use, modification, and redistribution.
    
    \item \textbf{Virtual Machine Images}: Vulnerable VMs (Metasploitable, DVWA) are provided by their respective projects with explicit permission for cybersecurity training and research purposes.
    
    \item \textbf{Documentation Ownership}: Per UPF regulations (Article 10), intellectual property of the TFG belongs to the student. Exploitation rights default to the student unless otherwise agreed with the tutor and institution.
    
    \item \textbf{Third-Party Content}: All references, citations, and external resources used in documentation follow IEEE citation standards, ensuring proper attribution and avoiding plagiarism.
\end{itemize}

\subsubsection{Ethical and Legal Considerations for Pentesting}

The project adheres to ethical hacking principles and legal boundaries:

\begin{itemize}
    \item \textbf{Controlled Environment}: All pentesting activities are conducted within isolated virtualized environments (EVE-NG sandbox), with no connection to production networks or systems outside the lab.
    
    \item \textbf{Authorized Targets}: Only intentionally vulnerable systems (Metasploitable, DVWA) designed explicitly for pentesting training are used as targets.
    
    \item \textbf{Educational Purpose}: The project is conducted under academic supervision (Pere Vidiella i Catalan, tutor) within the scope of the "Introduction to Cybersecurity" course curriculum, satisfying the educational exception for security research.
    
    \item \textbf{Data Privacy}: No personally identifiable information (PII) or sensitive data is processed during laboratory exercises. All scenarios use synthetic data or publicly available test datasets.
    
    \item \textbf{Responsible Disclosure}: In the unlikely event that previously unknown vulnerabilities are discovered in open-source software during testing, they will be reported responsibly to the respective maintainers following coordinated disclosure practices.
\end{itemize}

\subsubsection{Compliance with Academic Regulations}

The project follows Tecnocampus ESUPT TFG regulations (approved September 30, 2021):

\begin{itemize}
    \item \textbf{Plagiarism Prevention}: All work is original, properly cited, and verified using plagiarism detection tools. Adherence to Article 11 (Plagiarism) ensuring academic integrity.
    
    \item \textbf{Evaluation Criteria}: The project structure follows Chapter 9 (Evaluation) requirements, including continuous assessment (avantprojecte, interim report, final delivery) and tribunal defense.
    
    \item \textbf{Documentation Standards}: Formatting adheres to the "Guia per a la redacció i edició del TFG" (ESUPT guidelines), ensuring compliance with submission requirements.
\end{itemize}

\textbf{Conclusion}: The project demonstrates \textbf{full legal compliance} with intellectual property regulations, ethical hacking principles, and institutional academic standards.

\subsection{Gender and Diversity Perspective}

\subsubsection{Inclusive Language and Representation}

The project incorporates gender and diversity considerations:

\begin{itemize}
    \item \textbf{Inclusive Language}: All documentation (thesis, technical write-ups, student guides) uses gender-neutral language where possible, avoiding gender-biased assumptions.
    
    \item \textbf{Diverse Examples}: Laboratory scenarios and fictional user accounts employ diverse names and backgrounds, reflecting the multicultural and gender-diverse nature of the cybersecurity profession.
    
    \item \textbf{Accessibility}: Documentation is structured to be accessible to students with diverse learning needs, including clear formatting, descriptive headings, and optional use of dyslexia-friendly fonts (OpenDyslexic) in LaTeX template.
\end{itemize}

\subsubsection{Addressing the Gender Gap in Cybersecurity}

Cybersecurity remains a male-dominated field. This project contributes to bridging the gender gap:

\begin{itemize}
    \item \textbf{Accessible Entry Point}: Well-documented, structured laboratories lower barriers to entry for students traditionally underrepresented in technical fields, providing a supportive learning environment.
    
    \item \textbf{Ethical Framework}: Emphasizing the ethical aspects of cybersecurity (responsible disclosure, defensive security, societal impact) may appeal to a broader demographic beyond the stereotype of "hacker culture."
    
    \item \textbf{Collaborative Learning}: The design encourages collaborative problem-solving and knowledge sharing rather than competitive dynamics, fostering an inclusive learning culture.
\end{itemize}

\textbf{Conclusion}: The project demonstrates \textbf{awareness and commitment} to gender and diversity considerations, incorporating inclusive language, diverse representation, and accessibility measures that align with modern educational best practices.
