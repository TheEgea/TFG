% Plantilla LaTeX per a la redacció de Treballs de Fi de Grau (TFG)
% Autor: Eloi Egea Rada
%
% © 2025, Eloi Egea Rada
% Aquesta obra està sota una llicència Creative Commons Atribució-NoComercial-CompartirIgual 4.0 Internacional (CC BY-NC-SA 4.0).

\newglossaryentry{api_rest}{
    name={API REST},
    description={Interfície de Programació d'Aplicacions basada en el paradigma \textit{Representational State Transfer} (REST), que utilitza el protocol HTTP per facilitar la comunicació entre sistemes i permet realitzar operacions com consultar, crear, modificar i eliminar recursos de manera escalable i eficient}
}

% Afegir més entrades rellevants al TFG sobre Pentesting
\newglossaryentry{pentesting}{
    name={Pentesting},
    description={Assaig de penetració o \textit{penetration testing}, procés autoritzat de simulació d'atacs cibernètics per identificar vulnerabilitats de seguretat en sistemes, xarxes i aplicacions}
}

\newglossaryentry{eve_ng}{
    name={EVE-NG},
    description={\textit{Emulated Virtual Environment - Next Generation}, plataforma de simulació virtual que permet crear topologies de xarxa complexes amb diversos dispositius de xarxa per a propòsits educatius i de testeig}
}

\newglossaryentry{kali}{
    name={Kali Linux},
    description={Distribució Linux especialitzada en pentesting i auditoria de seguretat que inclou centenars d'eines de força bruta, sniffing, escanneig de vulnerabilitats i altres utilitats per a professionals de ciberseguretat}
}

\newglossaryentry{vulnerability}{
    name={Vulnerabilitat},
    description={Debilitat o defecte en un sistema, aplicació o infraestructura que pot ser explotada per un atacant per accedir-hi sense autorització, causar danys o comprometre la seguretat dels dades}
}

\newglossaryentry{exploit}{
    name={Exploit},
    description={Codi, tècnica o procediment que aprofita una vulnerabilitat específica per aconseguir accés no autoritzat o executar codi maliciós en un sistema objectiu}
}

\newglossaryentry{payload}{
    name={Payload},
    description={Component o dades que es transmeten com a part d'un atac, contenen el codi maliciós o les instruccions que l'atacant vol executar en el sistema comprès}
}

\newglossaryentry{reconnaissance}{
    name={Reconeixement},
    description={Fase inicial d'un assaig de penetració on l'atacant recopila informació sobre l'objectiu (sistemes, usuaris, xarxes, serveis) sense accés directe, típicament mitjançant tècniques passives}
}

\newglossaryentry{cvss}{
    name={CVSS},
    description={\textit{Common Vulnerability Scoring System}, sistema estàndard per a avaluar la gravetat de les vulnerabilitats mitjançant una puntuació numèrica que reflecteix la facilitat d'explotació i l'impacte potencial}
}

\newglossaryentry{owasp}{
    name={OWASP},
    description={\textit{Open Web Application Security Project}, comunitat internacional dedicada a millorar la seguretat del programari web, que publica recursos com el \textit{Top 10} de vulnerabilitats més crítigues}
}
