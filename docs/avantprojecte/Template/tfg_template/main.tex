% Plantilla LaTeX per a la redacció de Treballs de Fi de Grau (TFG)
% Autor: Eloi Egea Rada
%
% © 2025, Eloi Egea Rada
% Aquesta obra està sota una llicència Creative Commons Atribució-NoComercial-CompartirIgual 4.0 Internacional (CC BY-NC-SA 4.0). https://creativecommons.org/licenses/by-nc-sa/4.0/
% Plantilla per als Treballs de Final de Grau del Centre Universitari TecnoCampus. Fent referencia a la plantilla original d'Aitor Juanola.

\documentclass[a4paper, twoside, 12pt, openright]{report} % 'openright' assegura que cada capítol comença en una plana senar

% ============================================================================
% PREÀMBUL — Paquets i Configuració
% ============================================================================

\usepackage[utf8]{inputenc} % Codificació UTF-8
\usepackage{verbatim} % Comentaris multilínea
\usepackage{graphicx} % Per insertar imatges
\usepackage{float} % Posicionament d'imatges

\usepackage[english]{babel} % Idioma del document
\usepackage{csquotes}  % Recomanat per biblatex amb babel/english
\usepackage{hyperref}

\usepackage{fancyhdr} % Per personalitzar capçaleres i peus de pàgina
\usepackage{etoolbox} % Per canviar estils de numeració
\usepackage{geometry} % Per ajustar marges
%\usepackage{emptypage} % Per evitar capçaleres/peus a pàgines en blanc
\usepackage{hyperref} % Afegeix glossari a la taula de continguts
\usepackage{setspace} % Per ajustar l'interlineat
\usepackage{titlesec} % Paquet per personalitzar els títols
\usepackage{times} % Paquet per fonts Times New Roman
\usepackage[nottoc, numbib]{tocbibind} % Inclou bibliografia i índexs automàticament al ToC
\usepackage[nonumberlist, toc]{glossaries} % Paquet per a la generació del glossari
\usepackage[style=ieee, backend=biber]{biblatex} % Paquet per gestionar la bibliografia en format IEEE
\addbibresource{resources/references.bib} % El fitxer .bib amb les referències

% Carregar les definicions del glossari des d'un fitxer extern
% El fitxer real està a `docs/avantprojecte/Template/tfg_template/resources/glossary.tex` — usem una ruta relativa des de la plantilla
% Plantilla per als Treballs de Final de Grau del Centre Universitari TecnoCampus.
% Autor: Eloi Egea Rada
% Primera versió, publicada al Gener de 2025.

% © 2025, Eloi Egea Rada
% Aquesta obra està sota una llicència Creative Commons Reconeixement - No comercial (CC BY - NC) 4.0 Internacional (https://creativecommons.org/licenses/by-nc/4.0/)

\newglossaryentry{api_rest}{
    name={API REST},
    description={Interfície de Programació d'Aplicacions basada en el paradigma \textit{Representational State Transfer} (REST), que utilitza el protocol HTTP per facilitar la comunicació entre sistemes i permet realitzar operacions com consultar, crear, modificar i eliminar recursos de manera escalable i eficient}
}


% Aplicar l'estil personalitzat
\setglossarystyle{altlist}
\setacronymstyle{long-short}

\makeglossaries
% Reduir espai abans i després de cada capítol
\titleformat{\chapter}[block]{\normalfont\fontsize{18pt}{0pt}\bfseries}{\thechapter}{1em}{}
\titlespacing*{\chapter}{0pt}{0pt}{\baselineskip}

\titleformat{\section}[block]{\normalfont\fontsize{16pt}{0pt}\bfseries}{\thesection}{1em}{}
\titleformat{\subsection}[block]{\normalfont\fontsize{14pt}{0pt}\bfseries}{\thesubsection}{1em}{}

% Configuració de hiperenllaços
\hypersetup{
    colorlinks=true,
    linkcolor=black,
    citecolor=black,
    filecolor=magenta,      
    urlcolor=blue,
    pdftitle={Pentesting ètic en entorns virtualitzats amb EVE-NG},
    pdfauthor={Eloi Egea Rada},
    pdfpagemode=FullScreen
}

% Configuració dels marges
\geometry{top=3cm, bottom=2.54cm, left=3cm, right=2.54cm, bindingoffset=0.46cm}

\setlength{\headheight}{14.5pt} % Augmentar l'altura de la capçalera

\setlength{\parindent}{0pt} % Desactiva el sagnat dels paràgrafs

% Imatges
\graphicspath{{images/}}

% ============================================================================
% PORTADA I PRELIMINARS
% ============================================================================

% Configuració de capçaleres i peus de pàgina preliminars
\fancypagestyle{plain}{
    \fancyhf{} % Esborrem configuracions prèvies
    \fancyhead[RO,LE]{\fontsize{10}{12}\textit{\thepage}}
    \fancyhead[RE]{\fontsize{10}{12}\textit{Pentesting ètic en entorns virtualitzats amb EVE-NG}}
    \fancyfoot{}
}
\pagestyle{plain}

\title{
        \begin{figure}
            \centering
            \includegraphics[width=0.85\linewidth]{images/logo-tecnocampus.png}
        \end{figure}
        
        \fontsize{14pt}{0pt}\textbf{Grau en Enginyeria Informàtica de Gestió i Sistemes d'Informació} \\
        
        \vspace*{3cm}

        \textbf{Pentesting ètic en entorns virtualitzats amb EVE-NG} \\

        \vspace*{1.5cm}

        \textbf{Memòria}

        \vspace*{5cm}
}
\author{
\fontsize{14pt}{0pt}
\textbf{Eloi Egea Rada} \\
\textbf{Tutor: Pere Vidiella i Catalan}
}
\date{Gener 2026}

\fontsize{12pt}{0pt}
\begin{document}

% === PORTADA ===
\maketitle

\newpage
\thispagestyle{empty} % Pàgina en blanc sense capçalera ni peu
\null
\newpage

% ============================================================================
% ÍNDEXS I GLOSSARI
% ============================================================================

% Estils dels índexs i numeració en xifres romanes
\pagenumbering{Roman} % Índexs comencen amb números romans

\tableofcontents


% Portada o altres planes sense numeració
%\listoffigures

%\listoftables

% Definir interlineat de 1,5 espais i espaiat anterior de 12pt
\onehalfspacing % Interlineat de 1,5
\setlength{\parskip}{12pt} % Espaiat anterior als paràgrafs

\printglossary


\begin{comment}
*****************************************************************************
*                           IMPORTANT!!!                                    *
*   Les següents dues comandes (\newpage, \null) estàn destinades a crear   *
*   una pàgina buida, això és degut a que en realitzar el canvi de          *
*   numeració (de romana a aràbiga), en crear un nou \chapter, no detecta   *
*   automàticament si la última pàgina es parell o imparell, per tant, un   *
*   cop acabat el glossari, cal comprovar si acaba en pàgina parell o       *
*   imparell, en aquest últim cas, cal afegir manualment una pàgina         *
*   en blanc, doncs sino es pot donar el cas que el glossari acabi en       *
*   la pàgina IX, i la següent pàgina sigui la 1, tractades les dues com    *
*   imparells.                                                              *
*****************************************************************************
\end{comment}
\newpage % Afegir nova pàgina, només en cas que el glossari acabi en pàgina imparell
\null
\newpage % Comença una nova pàgina per assegurar que la numeració sigui consistent
% Configuració de capçaleres i peus de pàgina principals
\fancypagestyle{plain}{
    \fancyhf{} % Esborrem configuracions prèvies
    \fancyhead[RO,LE]{\fontsize{10}{12}\textit{\thepage}}
    \fancyhead[LO]{\fontsize{10}{12}\textit{\nouppercase{\textit{\leftmark}}}}
    \fancyhead[RE]{\fontsize{10}{12}\textit{Pentesting ètic en entorns virtualitzats amb EVE-NG - TFG}}
    \fancyfoot{}
}
\pagestyle{plain}
% Canvi a numeració aràbiga per al cos principal del document
\pagenumbering{arabic} % Comença numeració aràbiga


% ============================================================================
% COS PRINCIPAL DEL DOCUMENT
% ============================================================================


\chapter{Introducció i Context}

\section{Motivació i Justificació}

El pentesting ètic és una disciplina fonamental en la ciberseguretat moderna. Aquest TFG 
es dedica a desenvolupar laboratoris pràctics virtualitzats que permeten als estudiants 
adquirir competències pràctiques en pentesting mitjançant escenaris realistes i controlats 
dins de la plataforma EVE-NG.

\section{Estructura de la Memòria}

La present memòria es divideix en els següents capítols:

\begin{enumerate}
    \item \textbf{Dedicatoria}: A qui va dirigit el treball, agraïments i reflexió personal.
    \item \textbf{Introducció i Context}: Motivació i estructura general.
    \item \textbf{Objecte del TFG}: Descripció detallada del projecte.
    \item \textbf{Estat de l'Art}: Revisió bibliogràfica i plataformes existents.
    \item \textbf{Objectius i Abast}: Objectius específics amb KPIs mesurables.
    \item \textbf{Metodologia}: Procés de realització i desenvolupament.
    \item \textbf{Requeriments}: Requeriments tècnics i funcionals.
    \item \textbf{Estudi de Viabilitat}: Análisis de viabilitat tècnica i econòmica.
    \item \textbf{Resultats i Conclusions}: Lliurables i conclusions.
\end{enumerate}

\newpage
% ============================================================================ç
% CAPÍTOL 1: DEDICATÒRIA
% ============================================================================

%\section{Dedicatoria}
\section{Dedication}

%\subsection{A qui va dirigit aquest treball}
\subsection{To Whom This Work is Dedicated}

%Aquest treball va dirigit a tots aquells estudiants i professionals interessats en la ciberseguretat, i en particular en el pentesting ètic. Esperem que els laboratoris i escenaris presentats en aquest TFG siguin d'utilitat per a la seva formació i desenvolupament professional.
This work is dedicated to all students and professionals interested in cybersecurity, particularly in ethical pentesting.
 We hope that the labs and scenarios presented in this TFG will be useful for their training and professional development.
\subsection{Acknowledgments}

We would like to thank all the people who have contributed in any way to the completion of this work.
First, our tutor, Pere Vidiella i Catalan, for his guidance and support throughout the process.
We would also like to thank our colleagues and the cybersecurity community for sharing valuable knowledge and resources.
Finally, a special thanks to our families and friends for their unconditional support.
\subsection{Personal Reflection}
The completion of this Bachelor's Thesis has been an enriching experience that has allowed me to deepen my knowledge of cybersecurity and develop technical and project management skills.
I have learned the importance of practice in learning this discipline and how virtualized environments can facilitate this process.
This project has motivated me to continue exploring the field of cybersecurity and to contribute to the training of future professionals in this critical area.
\newpage


% ============================================================================
% CAPÍTOL 2: OBJECTE DEL TFG
% ============================================================================


% 01_objecte.tex
% Document: TFG Template 
% Idioma: Inglés
\section{Objective of the TFG}
\subsection{Project Description}

The objective of this Final Degree Project is to develop a reusable teaching package of cybersecurity practical labs using the EVE-NG (Emulated Virtual Environment - Next Generation) platform for the course "Introduction to Cybersecurity" of the Bachelor's Degree in Computer Engineering in Management and Information Systems.

The project consists of creating a structured set of four virtualized labs covering the main areas of ethical pentesting: reconnaissance and enumeration, web application vulnerabilities, traffic analysis and cryptography, and privilege escalation. Each lab will include complete network topologies with preconfigured virtual machines, automation scripts for scenario deployment and reset, and comprehensive technical documentation.

\subsection{Rationale and Justification}

\subsection{Educational Context and Identified Needs}

Cybersecurity training requires safe practical environments where students can experiment with pentesting techniques without risks to real systems. Currently, the "Introduction to Cybersecurity" course at Tecnocampus has virtualized infrastructure developed in previous projects, but lacks a structured, automated, and reusable teaching package that facilitates both teaching and autonomous learning.

\subsubsection{Added Value and Innovation}

This TFG adds value in multiple dimensions:

\begin{enumerate}
\item \textbf{Reusability and Scalability}: The teaching package is designed to be used in multiple editions of the course, reducing the preparation load for the teaching staff.
\item \textbf{Automation of the Life Cycle}: Deployment, reset, and validation scripts allow efficient lab management, minimizing time spent on administrative tasks.
\item \textbf{Self-Guided Learning}: Structured documentation and assessment rubrics facilitate autonomous learning for students.
\item \textbf{Professional Standards}: The scenarios reproduce real vulnerabilities and techniques used in the professional field of cybersecurity.
\end{enumerate}

\subsection{Expected Outcomes}

The implementation of this teaching package will enable:

\begin{itemize}
\item Improved student learning experiences through realistic practices
\item Reduced preparation and management time for teachers
\item Enhanced reproducibility and consistency in student assessment
\item Establishment of a solid foundation for future developments in cybersecurity training
\end{itemize}

\subsection{Reference Framework}

\subsubsection{Academic Context}

This project is framed within the field of computer engineering applied to education, combining technical competencies in:

\begin{itemize}
\item \textbf{Software Engineering}: Development of automation scripts and validation systems
\item \textbf{Network-{protocols, services,  Administration of Systems and Services}}: Design of network architectures and IT infrastructure management
\item \textbf{Cybersecurity}: Implementation of ethical pentesting techniques and vulnerability analysis
\item \textbf{Project Management}: Planning, development, and delivery of a complete educational product
\end{itemize}

Obviously, there's a lot of other competencies involved, but those are the main ones that contribute to the successful development and implementation of the teaching package.

\subsection{Professional Context}

In the professional context, this project aligns with the growing demand for skilled cybersecurity professionals who can operate in complex virtual environments. The use of EVE-NG as a training platform reflects industry trends towards virtualization and cloud-based solutions. Furthermore, the focus on ethical pentesting techniques prepares students for real-world challenges, ensuring they are equipped with the necessary skills to succeed in the cybersecurity field. 

Despite on that, the main professional context is the educational one, as the project is aimed at improving the teaching and learning experience in cybersecurity education. 

\subsubsection{Technological Context}

EVE-NG is chosen as the base platform due to its ability to manage complex topologies, its compatibility with multiple virtualized operating systems, and its suitability for educational environments that require flexibility and reproducibility. The project leverages modern virtualization technologies and automation tools to create an efficient and effective learning environment for students.

Moreover, the use of widely adopted cybersecurity tools such as Kali Linux, Metasploit Framework, Nmap and digital forensic analysis software ensures that students gain hands-on experience with industry-standard technologies. This technological context not only enhances the learning experience but also prepares students for future professional roles in cybersecurity.



% ============================================================================
% CAPÍTOL 3: ESTAT DE L'ART
% ============================================================================

%\section{Estat de l'Art}
\section{State of the Art}
\subsection{Context and Background}

% Secció pendent de desenvolupament

\subsection{Overview of the Main Theories}

% Secció pendent de desenvolupament

\subsection{Existing Technological Solutions}

% Secció pendent de desenvolupament

\subsection{Interpretation and Reflection on the Sources}

% Secció pendent de desenvolupament

\subsection{Definition of Needs}

% Secció pendent de desenvolupament

% ============================================================================
% CAPÍTOL 4: OBJECTIUS I ABAST
% ============================================================================

%\section{Objectius i Abast}
\section{Objectives and Scope}
\subsection{Main Objective}

%Desenvolupar, implementar i validar un paquet docent reutilitzable de laboratoris pràctics de ciberseguretat basat en EVE-NG, que permeti als estudiants de l'assignatura ``Introduction to Cybersecurity'' adquirir competències pràctiques en pentesting ètic mitjançant escenaris realistes i automatitzats.
Develop, implement, and validate a reusable teaching package of practical cybersecurity labs based on EVE-NG, enabling students in the "Introduction to Cybersecurity" course to acquire practical skills in ethical pentesting through realistic and automated scenarios.
\subsection{Specific Objectives (Measurable with KPIs)}

\subsubsection{OE1: Design and Implement EVE-NG Labs}

%\textbf{Descripció}: Crear quatre laboratoris temàtics funcionals i interconnectats.
\textbf{Description}: Create four functional and interconnected thematic labs.
\textbf{Measurable KPIs}:
\begin{itemize}
\item 4 fully functional .unl topologies (100\%)
\item Deployment time $\leq$ 2 minutes per lab (95\% of cases)
\item VM boot success rate $\geq$ 95\%
\item Compatibility with EVE-NG Community and Professional editions
\end{itemize}

\textbf{Deliverables}:
\begin{itemize}
\item 4 .unl files (topologies)
\item 8-12 optimized virtual machine images
\item Technical configuration documentation for each lab
\end{itemize}

\subsubsection{OE2: Develop Automation Scripts}

\textbf{Description}: Create a complete automation system for the lab lifecycle.

\textbf{Measurable KPIs}:
\begin{itemize}
\item 12 functional scripts (3 per lab: deploy, reset, validate)
\item Reset execution time $\leq$ 30 seconds per lab
\item Automatic validation coverage $\geq$ 80\% of critical components
\item Zero errors in 10 consecutive executions of the complete cycle
\end{itemize}

\textbf{Deliverables}:
\begin{itemize}
\item Deployment scripts (Bash/Python)
\item Automated reset scripts
\item Validation and status verification scripts
\end{itemize}

\subsubsection{OE3: Develop Structured Teaching Material}

\textbf{Description}: Develop comprehensive documentation for students and teachers.

\textbf{Measurable KPIs}:
\begin{itemize}
\item 4 user manuals (1 per lab), minimum 15 pages each
\item 4 assessment rubrics with quantifiable criteria
\item 1 teacher guide with deployment instructions
\item Initial setup time per teacher $\leq$ 1 hour
\end{itemize}

\textbf{Deliverables}:
\begin{itemize}
\item Student manual (English, markdown/PDF format)
\item Teacher guide with detailed instructions
\item Structured assessment rubrics
\item Technical README with requirements and instructions
\end{itemize}

\subsubsection{OE4: Validate Usability and Educational Effectiveness}

\textbf{Description}: Verify that the package meets educational and technical requirements.

\textbf{Measurable KPIs}:
\begin{itemize}
\item Average resolution time per lab: 90-120 minutes
\item Successful completion rate $\geq$ 85\% (pilot with 10 users)
\item User satisfaction score $\geq$ 4/5
\item Zero critical incidents in production environment
\end{itemize}

\textbf{Deliverables}:
\begin{itemize}
\item Pilot validation report
\item Usability and performance metrics
\item Implemented improvement recommendations
\end{itemize}

%\subsection{Definició del Client i Usuari Final}
\subsection{Definition of the Client and Final User}
%\subsubsection{Client Principal}
\subsection{Principal Client}
\begin{itemize}
\item \textbf{Responsible professor}: Pere Vidiella i Catalan
\item \textbf{Course}: Introduction to Cybersecurity (GEISI, Tecnocampus)
\item \textbf{Context}: Undergraduate teaching in practical cybersecurity
\end{itemize}

\subsubsection{Primary End Users}

\begin{itemize}
\item \textbf{GEISI Students}: 25-30 students per edition
\item \textbf{Profile}: Basic knowledge of networks and operating systems
\item \textbf{Objective}: Learn ethical pentesting in a practical and safe manner
\end{itemize}

\subsubsection{Secondary End Users}

\begin{itemize}
\item \textbf{Cybersecurity Faculty}: Other teachers interested in reusing the material
\item \textbf{Postgraduate Students}: Possible extensions of the material for advanced levels
\end{itemize}

\subsection{Potential Audience}

\subsubsection{Internal Scope (Tecnocampus)}

\begin{itemize}
\item Other courses related to cybersecurity
\item Research projects in computer security
\item Continuing education and professional certifications
\end{itemize}

\subsubsection{External Scope}

\begin{itemize}
\item Educational institutions with training in cybersecurity
\item Specialized vocational training centers
\item Companies with internal training programs in security
\end{itemize}

\subsection{KPIs and Performance Indicators}

\subsubsection{Technical Indicators}

\begin{table}[h!]
\centering
\begin{tabular}{|l|l|l|}
\hline
\textbf{Mètrica} & \textbf{Objectiu} & \textbf{Mètode de Mesura} \\
\hline
Lab deployment time & $\leq$ 2 min & Automated timing \\
\hline
VM boot success rate & $\geq$ 95\% & System logs + validation scripts \\
\hline
Full reset time & $\leq$ 30 seg & Scripts with timestamps \\
\hline
Scenario reproducibility & 100\% & Automated tests \\
\hline
\end{tabular}
\end{table}

\subsubsection{Educational Indicators}

\begin{table}[h!]
\centering
\begin{tabular}{|l|l|l|}
\hline
\textbf{Mètrica} & \textbf{Objectiu} & \textbf{Mètode de Mesura} \\
\hline
Lab resolution time & 90-120 min & Time tracking during pilot \\
\hline
Successful completion rate & $\geq$ 85\% & User progress tracking during pilot \\
\hline
User satisfaction & $\geq$ 4/5 & Post-use survey \\
\hline
Key concept understanding & $\geq$ 80\% correcte & Assessment questionnaire \\
\hline
\end{tabular}
\end{table}

\subsubsection{Quality Indicators}

\begin{table}[h!]
\centering
\begin{tabular}{|l|l|l|}
\hline
\textbf{Mètrica} & \textbf{Objectiu} & \textbf{Mètode de Mesura} \\
\hline
Documentation coverage & 100\% of features & Verification checklist \\
\hline
Critical errors & 0 & Exhaustive testing \\
\hline
EVE-NG version compatibility & 100\% & Tests in multiple environments \\
\hline
Interface usability & $\geq$ 4/5 & UX heuristic evaluation \\
\hline
\end{tabular}
\end{table}

\noindent \textbf{Note}: All KPIs will be measured during the pilot validation phase (April-May 2026) and documented in the final TFG report.

% ============================================================================
% CAPÍTOL 5: METODOLOGIA
% ============================================================================

\section{Methodology}

\subsection{Final Degree Project (TFG) Implementation Process}

% Section pending development

\subsection{Stages, Tasks and Milestones}

% Section pending development

\subsection{Information Search Strategies}

% Section pending development

\subsection{Product Development Methodology}

% Section pending development

% ============================================================================
% CAPÍTOL 6: REQUERIMENTS
% ============================================================================

\section{Functional Requirements}

\subsection{RF1: Design EVE-NG Topologies}

- Create 4 .unl files compatible with EVE-NG 
- Implement OWASP  and CIS  security principles

\subsection{RF2: Implement Vulnerabilities}

- Lab 1: Reconnaissance tools per PTES 
- Lab 2: OWASP Top 10  vulnerabilities
- Lab 3: Network analysis per NIST standards 
- Lab 4: Privilege escalation per CEH framework 

\subsection{RF3: Create Assessment Materials}

- Assessment aligned with GEISI learning outcomes 
- Rubrics based on CIS Controls 

\section{Technological Requirements}

\subsection{TR1: Virtualization}

- EVE-NG Platform 
- KVM Hypervisor
- Network simulation capabilities

\subsection{TR2: Tools}

- Kali Linux  (attacking platform)
- Metasploit Framework  (exploitation)
- Nmap  (reconnaissance)
- Wireshark  (network analysis)
- Burp Suite  (web testing)
- OWASP ZAP  (web scanning)

\subsection{TR3: Scripts}

- Bash scripting
- Python automation

\section{Non-Functional Requirements}

\subsection{NFR1: Performance}

- Deployment per EVE-NG specifications 
- Boot times optimized for lab workflow

\subsection{NFR2: Compliance}

- NIST principles 
- CIS Controls 
- OWASP methodologies 

\subsection{NFR3: Institutional Sustainability}

- Alignment with GEISI curriculum 
- Open-source tools to ensure long-term maintainability


% ============================================================================
% CAPÍTOL 7: ESTUDI DE VIABILITAT
% ============================================================================

%\section{Estudi de la Viabilitat del Projecte}
\section{Feasibility Study of the Project}

%\subsection{Planificació Initial}
\subsection{Initial Planning}

%\subsubsection{Definició de Tasques}
\subsubsection{Task Definition}

% Secció pendent de desenvolupament

%\subsubsection{Diagrama de Gantt}
\subsubsection{Gantt Chart}

% Secció pendent de desenvolupament

%\subsubsection{Camins Crítics}
\subsubsection{Critical Paths}

% Secció pendent de desenvolupament

%\subsection{Pressupost}
\subsection{Budget}

%\subsubsection{Cost de Recursos Humans}
\subsubsection{Cost of Human Resources}

% Secció pendent de desenvolupament

%\subsubsection{Cost de Recursos Hardware i Software}
\subsubsection{Cost of Hardware and Software Resources}

% Secció pendent de desenvolupament

%\subsubsection{Altres Costos}
\subsubsection{Other Costs}

% Secció pendent de desenvolupament

\subsection{Feasibility Analysis}

\subsubsection{Technical Feasibility}

% Secció pendent de desenvolupament

\subsubsection{Economic Feasibility}

% Secció pendent de desenvolupament

\subsubsection{Environmental Feasibility}


% Secció pendent de desenvolupament

\subsubsection{Legal Aspects}

% Secció pendent de desenvolupament

\subsubsection{Gender and Diversity Perspective}

% Secció pendent de desenvolupament

% ============================================================================
% CAPÍTOL 8: DESENVOLUPAMENT I IMPLEMENTACIÓ (PLACEHOLDER)
% ============================================================================

\chapter{Desenvolupament i Implementació}

\section{Arquitectura General}

\textit{En aquesta secció es descriurà l'arquitectura dels laboratoris, la configuració 
de les xarxes virtuals, i la implementació de les topologies EVE-NG.}

\section{Laboratori 1: Reconeixement i Enumeració}

\subsection{Objectius de Laboratori}

\begin{itemize}
    \item Familiarització amb eines de scanning (nmap, Nessus)
    \item Enumeració de serveis i vulnerabilitats
    \item Documentació de descobriments
\end{itemize}

\subsection{Topologia de Xarxa}

\textit{Diagrama de la xarxa i màquines virtuals que componen aquest laboratori.}

\section{Laboratori 2: Vulnerabilitats d'Aplicacions Web}

\subsection{OWASP Top 10}

\textit{Exploració de les vulnerabilitats més crítiques segons OWASP.}

\section{Laboratori 3: Anàlisi de Tràfic i Criptografia}

\textit{Anàlisi de comunicacions de xarxa amb Wireshark i tècniques de criptografia.}

\section{Laboratori 4: Escalada de Privilegis}

\textit{Tècniques d'escalada de privilegis a sistemes Unix i Windows.}

% ============================================================================
% CAPÍTOL 9: RESULTATS I CONCLUSIONS
% ============================================================================

\chapter{Resultats i Conclusions}

\section{Lliurables Entregats}

\begin{itemize}
    \item 4 topologies EVE-NG funcionals (.unl)
    \item 8-12 imatges de màquines virtuals optimitzades
    \item Scripts d'automatització per a desplegament
    \item Documentació tècnica completa per laboratori
    \item Guies pràctiques per a estudiants
\end{itemize}

\section{Conclusions}

El desenvolupament d'aquests laboratoris proporciona una base sòlida per a la formació 
en pentesting ètic dins del Grau en Enginyeria Informàtica.

\section{Recomanacions per a Treballs Futurs}

\begin{itemize}
    \item Integració amb altres plataformes educatives (Moodle)
    \item Ampliació amb laboratoris d'IoT i xarxes 5G
    \item Automatització de correccions i qualificacions
\end{itemize}

% ============================================================================
% ANEXOS
% ============================================================================

\appendix

\chapter{Annexo A: Infraestructura de Build LaTeX Reproducible}

\section{Configuració de Docker i TeX Live}

La memòria del present TFG s'ha desenvolupat usant una infraestructura de build reproducible 
basada en TeX Live 2023, latexmk i Docker Compose. Aquesta aproximació garanteix que tots 
els artefactes (PDF, índexs, glosaris) es generin de manera consistent.

\subsection{Hores Invertides en Infraestructura}

\begin{table}[H]
    \centering
    \begin{tabular}{|l|l|r|}
    \hline
    \textbf{Tasca} & \textbf{Descripció} & \textbf{Hores} \\
    \hline
    TeX Live + Docker & Setup de compilació & 2.5h \\
    latexmk + .latexmkrc & Build automatitzat & 1.5h \\
    Estructura modular (chapters/) & Organització de capítols & 3.5h \\
    Validació i testing & Compilació final & 0.5h \\
    \hline
    \textbf{SUBTOTAL} & & \textbf{8.0h} \\
    \hline
    \end{tabular}
\end{table}

\chapter{Annexo B: Infraestructura de Laboratori (Homelab)}

\section{Especificacions del Homelab}

\begin{table}[H]
    \centering
    \begin{tabular}{|l|l|r|}
    \hline
    \textbf{Component} & \textbf{Especificació} & \textbf{Hores Setup} \\
    \hline
    Host VMware/Hyper-V & CPU 8-core, RAM 32GB, SSD 500GB & 3.0h \\
    EVE-NG & Simulador de xarxes corporatiu & 4.0h \\
    Kali Linux x3 & Màquines de pentesting & 2.5h \\
    Windows Server & Sistema objectiu & 1.5h \\
    Xarxes virtuals & 3 VLANS, DMZ, backend & 2.0h \\
    \hline
    \textbf{SUBTOTAL} & & \textbf{13.0h} \\
    \hline
    \end{tabular}
\end{table}

\chapter{Annexo C: Control de Hores Total}

\begin{table}[H]
    \centering
    \begin{tabular}{|l|r|r|}
    \hline
    \textbf{Activitat} & \textbf{Hores} & \textbf{Percentatge} \\
    \hline
    Infraestructura LaTeX & 8.0h & 6\% \\
    Infraestructura Homelab & 13.0h & 10\% \\
    Investigació i Disseny & 20.0h & 15\% \\
    Implementació de Laboratoris & 50.0h & 37\% \\
    Documentació & 25.0h & 19\% \\
    Testing i Validació & 15.0h & 11\% \\
    Gestió de Projecte & 5.0h & 4\% \\
    \hline
    \textbf{TOTAL ESTIMAT} & \textbf{136.0h} & \textbf{100\%} \\
    \hline
    \end{tabular}
\end{table}

% ============================================================================
% BIBLIOGRAFIA
% ============================================================================

\cleardoublepage
\phantomsection
\addcontentsline{toc}{chapter}{Bibliografia}
\printbibliography




\begin{comment}

\chapter{Nom capítol}

Lorem ipsum dolor sit amet, consectetur adipiscing elit, sed do eiusmod tempor incididunt ut labore et dolore magna aliqua. Ut enim ad minim veniam, quis nostrud exercitation ullamco laboris nisi ut aliquip ex ea commodo consequat. Duis aute irure dolor in reprehenderit in voluptate velit esse cillum dolore eu fugiat nulla pariatur. Excepteur sint occaecat cupidatat non proident, sunt in culpa qui officia deserunt mollit anim id est laborum.

\section{Nom secció}

Això és un exemple de referència al glossari: \gls{api_rest}.

\subsection{Nom subsecció}

Això és un exemple de referència bibliogràfica: \cite{tcm_web_presentation}.
Això és un altre exemple de referència bibliogràfica: \cite{smith2019}.
\chapter{Figures i taules}

\section{Exemple figura}

\begin{figure}[H]
    \centering
    \includegraphics[width=1\linewidth]{images/logo-tecnocampus.png}
    \caption{Logo Tecnocampus}
    \label{fig:logo-tecnocampus}
\end{figure}

Això és un exemple de com citar una figura: \ref{fig:logo-tecnocampus}.

\section{Exemple taula}


\begin{table}[H]
    \centering
    \begin{tabular}{ccc}
        A & B & C\\
        D & E & F\\
        G & H & I\\
    \end{tabular}
    \caption{Exemple taula}
    \label{tab:my-table}
\end{table}

Això és un exemple de com citar una taula: \ref{tab:my-table}


\cleardoublepage % Garanteix que la bibliografia comenci en una pàgina imparell
\phantomsection % Marca la pàgina correcta per a la taula de continguts
\addcontentsline{toc}{chapter}{Bibliografia} % Afegeix la bibliografia al ToC
\printbibliography % Imprimeix la bibliografia

\end{document}

\end{comment}

\end{document}