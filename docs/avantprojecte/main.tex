% LaTeX template for Avantprojecte (TFG Initial Proposal)
% 
% Author: Eloi Egea Rada
% © 2025, Eloi Egea Rada
% Licensed under CC BY-NC-SA 4.0
% Template for Final Degree Projects at Centre Universitari TecnoCampus

\documentclass[a4paper, twoside, 12pt, openright]{report}

% ============================================================================
% PREAMBLE — Packages and configuration
% ============================================================================

% --- Typography with XeLaTeX/LuaLaTeX ---
\usepackage{fontspec}
\usepackage{verbatim}
\usepackage{graphicx}
\usepackage{float}
\usepackage[english]{babel}
\usepackage{csquotes}
\usepackage{eurosym}
\usepackage{fancyhdr}
\usepackage{etoolbox}
\usepackage{geometry}
\usepackage{setspace}
\usepackage{titlesec}
\usepackage[nottoc, numbib]{tocbibind}
\usepackage[nonumberlist, toc]{glossaries}
\usepackage[style=ieee, backend=biber]{biblatex}

\usepackage{fancyhdr} % Per personalitzar capçaleres i peus de pàgina
\usepackage{etoolbox} % Per canviar estils de numeració
\usepackage{geometry} % Per ajustar marges
\usepackage{setspace} % Per ajustar l'interlineat
\usepackage{titlesec} % Paquet per personalitzar els títols
\usepackage[nottoc, numbib]{tocbibind} % Inclou bibliografia i índexs automàticament al ToC
\usepackage[nonumberlist, toc]{glossaries} % Paquet per a la generació del glossari
\usepackage[style=ieee, backend=biber]{biblatex} % Paquet per gestionar la bibliografia en format IEEE
\addbibresource{../resources/references.bib} % El fitxer .bib amb les referències

% Complex tables for time tracking
\usepackage{array}
\usepackage{tabularx}

% Hyperlinks (load AFTER other packages)
\usepackage{hyperref}

% ============================================================================
% OPENDYSLEXIC FONTS SELECTION
% ============================================================================

\setmainfont{OpenDyslexic}[
    Path = /usr/share/fonts/opentype/opendyslexic/,
    Extension = .otf,
    UprightFont = *-Regular,
    BoldFont = *-Bold,
    ItalicFont = *-Italic,
    BoldItalicFont = *-BoldItalic
]

\setsansfont{OpenDyslexicAlta}[
    Path = /usr/share/fonts/opentype/opendyslexic/,
    Extension = .otf,
    UprightFont = *-Regular,
    BoldFont = *-Bold,
    ItalicFont = *-Italic,
    BoldItalicFont = *-BoldItalic
]

\setmonofont{OpenDyslexicMono}[
    Path = /usr/share/fonts/opentype/opendyslexic/,
    Extension = .otf,
    UprightFont = *-Regular
]

% ============================================================================
% LOAD GLOSSARY
% ============================================================================



% Load glossary definitions from an external file
% Plantilla per als Treballs de Final de Grau del Centre Universitari TecnoCampus.
% Autor: Eloi Egea Rada
% Primera versió, publicada al Gener de 2025.

% © 2025, Eloi Egea Rada
% Aquesta obra està sota una llicència Creative Commons Reconeixement - No comercial (CC BY - NC) 4.0 Internacional (https://creativecommons.org/licenses/by-nc/4.0/)

\newglossaryentry{api_rest}{
    name={API REST},
    description={Interfície de Programació d'Aplicacions basada en el paradigma \textit{Representational State Transfer} (REST), que utilitza el protocol HTTP per facilitar la comunicació entre sistemes i permet realitzar operacions com consultar, crear, modificar i eliminar recursos de manera escalable i eficient}
}

\setglossarystyle{altlist}
\setacronymstyle{long-short}
\makeglossaries



% Apply glossary and acronym styles
\setglossarystyle{altlist}
\setacronymstyle{long-short}
\makeglossaries

% ============================================================================
% TITLE FORMATTING
% ============================================================================

\titleformat{\chapter}[block]{\normalfont\fontsize{18pt}{0pt}\bfseries}{\thechapter}{1em}{}
\titlespacing*{\chapter}{0pt}{0pt}{\baselineskip}

\titleformat{\section}[block]{\normalfont\fontsize{16pt}{0pt}\bfseries}{\thesection}{1em}{}
\titleformat{\subsection}[block]{\normalfont\fontsize{14pt}{0pt}\bfseries}{\thesubsection}{1em}{}

% ============================================================================
% HYPERLINK CONFIGURATION
% ============================================================================

\hypersetup{
    colorlinks=true,
    linkcolor=black,
    citecolor=black,
    filecolor=magenta,
    urlcolor=blue,
    pdftitle={Avantprojecte: Pentesting ètic en entorns virtualitzats amb EVE-NG},
    pdfauthor={Eloi Egea Rada},
    pdfpagemode=FullScreen
}

% ============================================================================
% PAGE MARGINS CONFIGURATION
% ============================================================================

\geometry{top=3cm, bottom=2.54cm, left=3cm, right=2.54cm, bindingoffset=0.46cm}
\setlength{\headheight}{14.5pt}
\setlength{\parindent}{0pt}


\setlength{\headheight}{14.5pt} % Augmentar l'altura de la capçalera


\setlength{\parindent}{0pt} % Desactiva el sagnat dels paràgrafs


% Imatges
\graphicspath{{../images/}}


% ============================================================================
% COVER PAGE & PRELIMINARIES
% ============================================================================

\fancypagestyle{plain}{
    \fancyhf{}
    \fancyhead[RO,LE]{\fontsize{10}{12}\textit{\thepage}}
    \fancyhead[RE]{\fontsize{10}{12}\textit{Avantprojecte TFG — Pentesting in virtualized environments}}
    \fancyfoot{}
}

\pagestyle{plain}

\title{
        \begin{figure}
            \centering
            \includegraphics[width=0.85\linewidth]{../images/logo-tecnocampus.png}
        \end{figure}
        
        \fontsize{14pt}{0pt}\textbf{Bachelor in Computer Engineering — Management and Information Systems} \\
        
        \vspace*{3cm}


        \textbf{Ethical Pentesting in Virtualized Environments with EVE-NG} \\


        \vspace*{1.5cm}


        \textbf{Final Degree Project Report}


        \vspace*{5cm}
}

\author{
\fontsize{14pt}{0pt}
	\textbf{Eloi Egea Rada} \\
	\textbf{Supervisor: Pere Vidiella i Catalan}
}

\date{December 2025}

\begin{document}

% === COVER PAGE ===
\maketitle

\newpage
\thispagestyle{empty}
\null
\newpage

% ============================================================================
% TABLE OF CONTENTS & PRELIMINARY PAGES
% ============================================================================

\pagenumbering{Roman}

\tableofcontents

\onehalfspacing
\setlength{\parskip}{12pt}

\printglossary

\newpage
\null
\newpage

% ============================================================================
% MAIN CONTENT (1-6 Chapters only — no development/results)
% ============================================================================

\fancypagestyle{plain}{
    \fancyhf{}
    \fancyhead[RO,LE]{\fontsize{10}{12}\textit{\thepage}}
    \fancyhead[LO]{\fontsize{10}{12}\textit{\nouppercase{\textit{\leftmark}}}}
    \fancyhead[RE]{\fontsize{10}{12}\textit{Avantprojecte TFG — Pentesting in virtualized environments}}
    \fancyfoot{}
}

\pagestyle{plain}

\pagenumbering{arabic}

% === CHAPTER 1: OBJECT ===

% 01_objecte.tex
% Document: TFG Template 
% Idioma: Inglés
\section{Objective of the TFG}
\subsection{Project Description}

The objective of this Final Degree Project is to develop a reusable teaching package of cybersecurity practical labs using the EVE-NG (Emulated Virtual Environment - Next Generation) platform for the course "Introduction to Cybersecurity" of the Bachelor's Degree in Computer Engineering in Management and Information Systems.

The project consists of creating a structured set of four virtualized labs covering the main areas of ethical pentesting: reconnaissance and enumeration, web application vulnerabilities, traffic analysis and cryptography, and privilege escalation. Each lab will include complete network topologies with preconfigured virtual machines, automation scripts for scenario deployment and reset, and comprehensive technical documentation.

\subsection{Rationale and Justification}

\subsection{Educational Context and Identified Needs}

Cybersecurity training requires safe practical environments where students can experiment with pentesting techniques without risks to real systems. Currently, the "Introduction to Cybersecurity" course at Tecnocampus has virtualized infrastructure developed in previous projects, but lacks a structured, automated, and reusable teaching package that facilitates both teaching and autonomous learning.

\subsubsection{Added Value and Innovation}

This TFG adds value in multiple dimensions:

\begin{enumerate}
\item \textbf{Reusability and Scalability}: The teaching package is designed to be used in multiple editions of the course, reducing the preparation load for the teaching staff.
\item \textbf{Automation of the Life Cycle}: Deployment, reset, and validation scripts allow efficient lab management, minimizing time spent on administrative tasks.
\item \textbf{Self-Guided Learning}: Structured documentation and assessment rubrics facilitate autonomous learning for students.
\item \textbf{Professional Standards}: The scenarios reproduce real vulnerabilities and techniques used in the professional field of cybersecurity.
\end{enumerate}

\subsection{Expected Outcomes}

The implementation of this teaching package will enable:

\begin{itemize}
\item Improved student learning experiences through realistic practices
\item Reduced preparation and management time for teachers
\item Enhanced reproducibility and consistency in student assessment
\item Establishment of a solid foundation for future developments in cybersecurity training
\end{itemize}

\subsection{Reference Framework}

\subsubsection{Academic Context}

This project is framed within the field of computer engineering applied to education, combining technical competencies in:

\begin{itemize}
\item \textbf{Software Engineering}: Development of automation scripts and validation systems
\item \textbf{Network-{protocols, services,  Administration of Systems and Services}}: Design of network architectures and IT infrastructure management
\item \textbf{Cybersecurity}: Implementation of ethical pentesting techniques and vulnerability analysis
\item \textbf{Project Management}: Planning, development, and delivery of a complete educational product
\end{itemize}

Obviously, there's a lot of other competencies involved, but those are the main ones that contribute to the successful development and implementation of the teaching package.

\subsection{Professional Context}

In the professional context, this project aligns with the growing demand for skilled cybersecurity professionals who can operate in complex virtual environments. The use of EVE-NG as a training platform reflects industry trends towards virtualization and cloud-based solutions. Furthermore, the focus on ethical pentesting techniques prepares students for real-world challenges, ensuring they are equipped with the necessary skills to succeed in the cybersecurity field. 

Despite on that, the main professional context is the educational one, as the project is aimed at improving the teaching and learning experience in cybersecurity education. 

\subsubsection{Technological Context}

EVE-NG is chosen as the base platform due to its ability to manage complex topologies, its compatibility with multiple virtualized operating systems, and its suitability for educational environments that require flexibility and reproducibility. The project leverages modern virtualization technologies and automation tools to create an efficient and effective learning environment for students.

Moreover, the use of widely adopted cybersecurity tools such as Kali Linux, Metasploit Framework, Nmap and digital forensic analysis software ensures that students gain hands-on experience with industry-standard technologies. This technological context not only enhances the learning experience but also prepares students for future professional roles in cybersecurity.



% === CHAPTER 2: STATE OF THE ART ===
%\section{Estat de l'Art}
\section{State of the Art}
\subsection{Context and Background}

% Secció pendent de desenvolupament

\subsection{Overview of the Main Theories}

% Secció pendent de desenvolupament

\subsection{Existing Technological Solutions}

% Secció pendent de desenvolupament

\subsection{Interpretation and Reflection on the Sources}

% Secció pendent de desenvolupament

\subsection{Definition of Needs}

% Secció pendent de desenvolupament

% === CHAPTER 3: OBJECTIVES ===
%\section{Objectius i Abast}
\section{Objectives and Scope}
\subsection{Main Objective}

%Desenvolupar, implementar i validar un paquet docent reutilitzable de laboratoris pràctics de ciberseguretat basat en EVE-NG, que permeti als estudiants de l'assignatura ``Introduction to Cybersecurity'' adquirir competències pràctiques en pentesting ètic mitjançant escenaris realistes i automatitzats.
Develop, implement, and validate a reusable teaching package of practical cybersecurity labs based on EVE-NG, enabling students in the "Introduction to Cybersecurity" course to acquire practical skills in ethical pentesting through realistic and automated scenarios.
\subsection{Specific Objectives (Measurable with KPIs)}

\subsubsection{OE1: Design and Implement EVE-NG Labs}

%\textbf{Descripció}: Crear quatre laboratoris temàtics funcionals i interconnectats.
\textbf{Description}: Create four functional and interconnected thematic labs.
\textbf{Measurable KPIs}:
\begin{itemize}
\item 4 fully functional .unl topologies (100\%)
\item Deployment time $\leq$ 2 minutes per lab (95\% of cases)
\item VM boot success rate $\geq$ 95\%
\item Compatibility with EVE-NG Community and Professional editions
\end{itemize}

\textbf{Deliverables}:
\begin{itemize}
\item 4 .unl files (topologies)
\item 8-12 optimized virtual machine images
\item Technical configuration documentation for each lab
\end{itemize}

\subsubsection{OE2: Develop Automation Scripts}

\textbf{Description}: Create a complete automation system for the lab lifecycle.

\textbf{Measurable KPIs}:
\begin{itemize}
\item 12 functional scripts (3 per lab: deploy, reset, validate)
\item Reset execution time $\leq$ 30 seconds per lab
\item Automatic validation coverage $\geq$ 80\% of critical components
\item Zero errors in 10 consecutive executions of the complete cycle
\end{itemize}

\textbf{Deliverables}:
\begin{itemize}
\item Deployment scripts (Bash/Python)
\item Automated reset scripts
\item Validation and status verification scripts
\end{itemize}

\subsubsection{OE3: Develop Structured Teaching Material}

\textbf{Description}: Develop comprehensive documentation for students and teachers.

\textbf{Measurable KPIs}:
\begin{itemize}
\item 4 user manuals (1 per lab), minimum 15 pages each
\item 4 assessment rubrics with quantifiable criteria
\item 1 teacher guide with deployment instructions
\item Initial setup time per teacher $\leq$ 1 hour
\end{itemize}

\textbf{Deliverables}:
\begin{itemize}
\item Student manual (English, markdown/PDF format)
\item Teacher guide with detailed instructions
\item Structured assessment rubrics
\item Technical README with requirements and instructions
\end{itemize}

\subsubsection{OE4: Validate Usability and Educational Effectiveness}

\textbf{Description}: Verify that the package meets educational and technical requirements.

\textbf{Measurable KPIs}:
\begin{itemize}
\item Average resolution time per lab: 90-120 minutes
\item Successful completion rate $\geq$ 85\% (pilot with 10 users)
\item User satisfaction score $\geq$ 4/5
\item Zero critical incidents in production environment
\end{itemize}

\textbf{Deliverables}:
\begin{itemize}
\item Pilot validation report
\item Usability and performance metrics
\item Implemented improvement recommendations
\end{itemize}

%\subsection{Definició del Client i Usuari Final}
\subsection{Definition of the Client and Final User}
%\subsubsection{Client Principal}
\subsection{Principal Client}
\begin{itemize}
\item \textbf{Responsible professor}: Pere Vidiella i Catalan
\item \textbf{Course}: Introduction to Cybersecurity (GEISI, Tecnocampus)
\item \textbf{Context}: Undergraduate teaching in practical cybersecurity
\end{itemize}

\subsubsection{Primary End Users}

\begin{itemize}
\item \textbf{GEISI Students}: 25-30 students per edition
\item \textbf{Profile}: Basic knowledge of networks and operating systems
\item \textbf{Objective}: Learn ethical pentesting in a practical and safe manner
\end{itemize}

\subsubsection{Secondary End Users}

\begin{itemize}
\item \textbf{Cybersecurity Faculty}: Other teachers interested in reusing the material
\item \textbf{Postgraduate Students}: Possible extensions of the material for advanced levels
\end{itemize}

\subsection{Potential Audience}

\subsubsection{Internal Scope (Tecnocampus)}

\begin{itemize}
\item Other courses related to cybersecurity
\item Research projects in computer security
\item Continuing education and professional certifications
\end{itemize}

\subsubsection{External Scope}

\begin{itemize}
\item Educational institutions with training in cybersecurity
\item Specialized vocational training centers
\item Companies with internal training programs in security
\end{itemize}

\subsection{KPIs and Performance Indicators}

\subsubsection{Technical Indicators}

\begin{table}[h!]
\centering
\begin{tabular}{|l|l|l|}
\hline
\textbf{Mètrica} & \textbf{Objectiu} & \textbf{Mètode de Mesura} \\
\hline
Lab deployment time & $\leq$ 2 min & Automated timing \\
\hline
VM boot success rate & $\geq$ 95\% & System logs + validation scripts \\
\hline
Full reset time & $\leq$ 30 seg & Scripts with timestamps \\
\hline
Scenario reproducibility & 100\% & Automated tests \\
\hline
\end{tabular}
\end{table}

\subsubsection{Educational Indicators}

\begin{table}[h!]
\centering
\begin{tabular}{|l|l|l|}
\hline
\textbf{Mètrica} & \textbf{Objectiu} & \textbf{Mètode de Mesura} \\
\hline
Lab resolution time & 90-120 min & Time tracking during pilot \\
\hline
Successful completion rate & $\geq$ 85\% & User progress tracking during pilot \\
\hline
User satisfaction & $\geq$ 4/5 & Post-use survey \\
\hline
Key concept understanding & $\geq$ 80\% correcte & Assessment questionnaire \\
\hline
\end{tabular}
\end{table}

\subsubsection{Quality Indicators}

\begin{table}[h!]
\centering
\begin{tabular}{|l|l|l|}
\hline
\textbf{Mètrica} & \textbf{Objectiu} & \textbf{Mètode de Mesura} \\
\hline
Documentation coverage & 100\% of features & Verification checklist \\
\hline
Critical errors & 0 & Exhaustive testing \\
\hline
EVE-NG version compatibility & 100\% & Tests in multiple environments \\
\hline
Interface usability & $\geq$ 4/5 & UX heuristic evaluation \\
\hline
\end{tabular}
\end{table}

\noindent \textbf{Note}: All KPIs will be measured during the pilot validation phase (April-May 2026) and documented in the final TFG report.

% === CHAPTER 4: METHODOLOGY ===
\section{Methodology}

\subsection{Final Degree Project (TFG) Implementation Process}

% Section pending development

\subsection{Stages, Tasks and Milestones}

% Section pending development

\subsection{Information Search Strategies}

% Section pending development

\subsection{Product Development Methodology}

% Section pending development

% === CHAPTER 5: REQUIREMENTS ===
\section{Functional Requirements}

\subsection{RF1: Design EVE-NG Topologies}

- Create 4 .unl files compatible with EVE-NG 
- Implement OWASP  and CIS  security principles

\subsection{RF2: Implement Vulnerabilities}

- Lab 1: Reconnaissance tools per PTES 
- Lab 2: OWASP Top 10  vulnerabilities
- Lab 3: Network analysis per NIST standards 
- Lab 4: Privilege escalation per CEH framework 

\subsection{RF3: Create Assessment Materials}

- Assessment aligned with GEISI learning outcomes 
- Rubrics based on CIS Controls 

\section{Technological Requirements}

\subsection{TR1: Virtualization}

- EVE-NG Platform 
- KVM Hypervisor
- Network simulation capabilities

\subsection{TR2: Tools}

- Kali Linux  (attacking platform)
- Metasploit Framework  (exploitation)
- Nmap  (reconnaissance)
- Wireshark  (network analysis)
- Burp Suite  (web testing)
- OWASP ZAP  (web scanning)

\subsection{TR3: Scripts}

- Bash scripting
- Python automation

\section{Non-Functional Requirements}

\subsection{NFR1: Performance}

- Deployment per EVE-NG specifications 
- Boot times optimized for lab workflow

\subsection{NFR2: Compliance}

- NIST principles 
- CIS Controls 
- OWASP methodologies 

\subsection{NFR3: Institutional Sustainability}

- Alignment with GEISI curriculum 
- Open-source tools to ensure long-term maintainability


% === CHAPTER 6: FEASIBILITY STUDY (Complete) ===
%\section{Estudi de la Viabilitat del Projecte}
\section{Feasibility Study of the Project}

%\subsection{Planificació Initial}
\subsection{Initial Planning}

%\subsubsection{Definició de Tasques}
\subsubsection{Task Definition}

% Secció pendent de desenvolupament

%\subsubsection{Diagrama de Gantt}
\subsubsection{Gantt Chart}

% Secció pendent de desenvolupament

%\subsubsection{Camins Crítics}
\subsubsection{Critical Paths}

% Secció pendent de desenvolupament

%\subsection{Pressupost}
\subsection{Budget}

%\subsubsection{Cost de Recursos Humans}
\subsubsection{Cost of Human Resources}

% Secció pendent de desenvolupament

%\subsubsection{Cost de Recursos Hardware i Software}
\subsubsection{Cost of Hardware and Software Resources}

% Secció pendent de desenvolupament

%\subsubsection{Altres Costos}
\subsubsection{Other Costs}

% Secció pendent de desenvolupament

\subsection{Feasibility Analysis}

\subsubsection{Technical Feasibility}

% Secció pendent de desenvolupament

\subsubsection{Economic Feasibility}

% Secció pendent de desenvolupament

\subsubsection{Environmental Feasibility}


% Secció pendent de desenvolupament

\subsubsection{Legal Aspects}

% Secció pendent de desenvolupament

\subsubsection{Gender and Diversity Perspective}

% Secció pendent de desenvolupament

% ============================================================================
% BIBLIOGRAPHY
% ============================================================================



%\chapter{Introducció i Context}
\chapter{Introduction and Context}


%\section{Motivació i Justificació}
\section{Motivation and Justification}

Ethical hacking, or pentesting, is a critical skill in today's cybersecurity landscape. With the increasing 
reliance on virtualized environments, there is a growing need for practical training platforms that simulate 
real-world scenarios. While platforms such as HackTheBox and TryHackMe have become increasingly popular for 
cybersecurity training, they often focus on isolated challenges and gamified scenarios that may not directly 
reflect the practical application of computer science fundamentals taught in a degree program. EVE-NG 
(Emulated Virtual Environment Next Generation) offers a complementary and more customizable approach, 
providing a robust solution for creating controlled, hands-on learning environments.

This project aims to develop a series of practical laboratories using EVE-NG that explicitly bridge the 
foundational concepts covered throughout the degree program---including databases, network protocols, and 
operating system internals---with real-world security vulnerabilities and pentesting techniques. By 
demonstrating how these academic concepts can be exploited, students gain a deeper appreciation for 
defensive security practices and the critical importance of secure system design. This approach transforms 
abstract knowledge into hands-on experience where students can observe the direct consequences of security 
misconfigurations and understand why rigorous cybersecurity practices are essential.

Furthermore, by leveraging EVE-NG, this project seeks to fill a gap in comprehensive educational resources 
that integrate multiple pentesting tools and attack scenarios into a cohesive learning environment aligned 
with the Computer Engineering curriculum. This reproducible, risk-free, and fully customizable approach 
provides students with the opportunity to practice ethical hacking in a controlled setting, progressively 
building upon prior learning while preparing them for the professional responsibility of securing the 
systems they will design and manage in their careers.




%\section{Estructura de la Memòria}
\section{Structure of the Report}


This report is divided into the following chapters:


\begin{enumerate}
    \item \textbf{Dedicatoria}: To whom the work is dedicated, acknowledgments, and personal reflection.
    \item \textbf{Introduction and Context}: Motivation and overall structure.
    \item \textbf{Object of the TFG}: Detailed description of the project.
    \item \textbf{State of the Art}: Literature review and existing platforms.
    \item \textbf{Objectives and Scope}: Specific objectives with measurable KPIs.
    \item \textbf{Methodology}: Process of realization and development.
    \item \textbf{Requirements}: Technical and functional requirements.
    \item \textbf{Feasibility Study}: Analysis of technical and economic feasibility.
    \item \textbf{Results and Conclusions}: Deliverables and conclusions.
\end{enumerate}

\newpage
% ============================================================================
% CAPÍTOL 1: DEDICATÒRIA
% ============================================================================


%\section{Dedicatoria}
\section{Dedication}

%\subsection{A qui va dirigit aquest treball}
\subsection{To Whom This Work is Dedicated}

%Aquest treball va dirigit a tots aquells estudiants i professionals interessats en la ciberseguretat, i en particular en el pentesting ètic. Esperem que els laboratoris i escenaris presentats en aquest TFG siguin d'utilitat per a la seva formació i desenvolupament professional.
This work is dedicated to all students and professionals interested in cybersecurity, particularly in ethical pentesting.
 We hope that the labs and scenarios presented in this TFG will be useful for their training and professional development.
\subsection{Acknowledgments}

We would like to thank all the people who have contributed in any way to the completion of this work.
First, our tutor, Pere Vidiella i Catalan, for his guidance and support throughout the process.
We would also like to thank our colleagues and the cybersecurity community for sharing valuable knowledge and resources.
Finally, a special thanks to our families and friends for their unconditional support.
\subsection{Personal Reflection}
The completion of this Bachelor's Thesis has been an enriching experience that has allowed me to deepen my knowledge of cybersecurity and develop technical and project management skills.
I have learned the importance of practice in learning this discipline and how virtualized environments can facilitate this process.
This project has motivated me to continue exploring the field of cybersecurity and to contribute to the training of future professionals in this critical area.
\newpage



% ============================================================================
% CAPÍTOL 2: OBJECTE DEL TFG
% ============================================================================



% 01_objecte.tex
% Document: TFG Template 
% Idioma: Inglés
\section{Objective of the TFG}
\subsection{Project Description}

The objective of this Final Degree Project is to develop a reusable teaching package of cybersecurity practical labs using the EVE-NG (Emulated Virtual Environment - Next Generation) platform for the course "Introduction to Cybersecurity" of the Bachelor's Degree in Computer Engineering in Management and Information Systems.

The project consists of creating a structured set of four virtualized labs covering the main areas of ethical pentesting: reconnaissance and enumeration, web application vulnerabilities, traffic analysis and cryptography, and privilege escalation. Each lab will include complete network topologies with preconfigured virtual machines, automation scripts for scenario deployment and reset, and comprehensive technical documentation.

\subsection{Rationale and Justification}

\subsection{Educational Context and Identified Needs}

Cybersecurity training requires safe practical environments where students can experiment with pentesting techniques without risks to real systems. Currently, the "Introduction to Cybersecurity" course at Tecnocampus has virtualized infrastructure developed in previous projects, but lacks a structured, automated, and reusable teaching package that facilitates both teaching and autonomous learning.

\subsubsection{Added Value and Innovation}

This TFG adds value in multiple dimensions:

\begin{enumerate}
\item \textbf{Reusability and Scalability}: The teaching package is designed to be used in multiple editions of the course, reducing the preparation load for the teaching staff.
\item \textbf{Automation of the Life Cycle}: Deployment, reset, and validation scripts allow efficient lab management, minimizing time spent on administrative tasks.
\item \textbf{Self-Guided Learning}: Structured documentation and assessment rubrics facilitate autonomous learning for students.
\item \textbf{Professional Standards}: The scenarios reproduce real vulnerabilities and techniques used in the professional field of cybersecurity.
\end{enumerate}

\subsection{Expected Outcomes}

The implementation of this teaching package will enable:

\begin{itemize}
\item Improved student learning experiences through realistic practices
\item Reduced preparation and management time for teachers
\item Enhanced reproducibility and consistency in student assessment
\item Establishment of a solid foundation for future developments in cybersecurity training
\end{itemize}

\subsection{Reference Framework}

\subsubsection{Academic Context}

This project is framed within the field of computer engineering applied to education, combining technical competencies in:

\begin{itemize}
\item \textbf{Software Engineering}: Development of automation scripts and validation systems
\item \textbf{Network-{protocols, services,  Administration of Systems and Services}}: Design of network architectures and IT infrastructure management
\item \textbf{Cybersecurity}: Implementation of ethical pentesting techniques and vulnerability analysis
\item \textbf{Project Management}: Planning, development, and delivery of a complete educational product
\end{itemize}

Obviously, there's a lot of other competencies involved, but those are the main ones that contribute to the successful development and implementation of the teaching package.

\subsection{Professional Context}

In the professional context, this project aligns with the growing demand for skilled cybersecurity professionals who can operate in complex virtual environments. The use of EVE-NG as a training platform reflects industry trends towards virtualization and cloud-based solutions. Furthermore, the focus on ethical pentesting techniques prepares students for real-world challenges, ensuring they are equipped with the necessary skills to succeed in the cybersecurity field. 

Despite on that, the main professional context is the educational one, as the project is aimed at improving the teaching and learning experience in cybersecurity education. 

\subsubsection{Technological Context}

EVE-NG is chosen as the base platform due to its ability to manage complex topologies, its compatibility with multiple virtualized operating systems, and its suitability for educational environments that require flexibility and reproducibility. The project leverages modern virtualization technologies and automation tools to create an efficient and effective learning environment for students.

Moreover, the use of widely adopted cybersecurity tools such as Kali Linux, Metasploit Framework, Nmap and digital forensic analysis software ensures that students gain hands-on experience with industry-standard technologies. This technological context not only enhances the learning experience but also prepares students for future professional roles in cybersecurity.




% ============================================================================
% CAPÍTOL 3: ESTAT DE L'ART
% ============================================================================


%\section{Estat de l'Art}
\section{State of the Art}
\subsection{Context and Background}

% Secció pendent de desenvolupament

\subsection{Overview of the Main Theories}

% Secció pendent de desenvolupament

\subsection{Existing Technological Solutions}

% Secció pendent de desenvolupament

\subsection{Interpretation and Reflection on the Sources}

% Secció pendent de desenvolupament

\subsection{Definition of Needs}

% Secció pendent de desenvolupament


% ============================================================================
% CAPÍTOL 4: OBJECTIUS I ABAST
% ============================================================================


%\section{Objectius i Abast}
\section{Objectives and Scope}
\subsection{Main Objective}

%Desenvolupar, implementar i validar un paquet docent reutilitzable de laboratoris pràctics de ciberseguretat basat en EVE-NG, que permeti als estudiants de l'assignatura ``Introduction to Cybersecurity'' adquirir competències pràctiques en pentesting ètic mitjançant escenaris realistes i automatitzats.
Develop, implement, and validate a reusable teaching package of practical cybersecurity labs based on EVE-NG, enabling students in the "Introduction to Cybersecurity" course to acquire practical skills in ethical pentesting through realistic and automated scenarios.
\subsection{Specific Objectives (Measurable with KPIs)}

\subsubsection{OE1: Design and Implement EVE-NG Labs}

%\textbf{Descripció}: Crear quatre laboratoris temàtics funcionals i interconnectats.
\textbf{Description}: Create four functional and interconnected thematic labs.
\textbf{Measurable KPIs}:
\begin{itemize}
\item 4 fully functional .unl topologies (100\%)
\item Deployment time $\leq$ 2 minutes per lab (95\% of cases)
\item VM boot success rate $\geq$ 95\%
\item Compatibility with EVE-NG Community and Professional editions
\end{itemize}

\textbf{Deliverables}:
\begin{itemize}
\item 4 .unl files (topologies)
\item 8-12 optimized virtual machine images
\item Technical configuration documentation for each lab
\end{itemize}

\subsubsection{OE2: Develop Automation Scripts}

\textbf{Description}: Create a complete automation system for the lab lifecycle.

\textbf{Measurable KPIs}:
\begin{itemize}
\item 12 functional scripts (3 per lab: deploy, reset, validate)
\item Reset execution time $\leq$ 30 seconds per lab
\item Automatic validation coverage $\geq$ 80\% of critical components
\item Zero errors in 10 consecutive executions of the complete cycle
\end{itemize}

\textbf{Deliverables}:
\begin{itemize}
\item Deployment scripts (Bash/Python)
\item Automated reset scripts
\item Validation and status verification scripts
\end{itemize}

\subsubsection{OE3: Develop Structured Teaching Material}

\textbf{Description}: Develop comprehensive documentation for students and teachers.

\textbf{Measurable KPIs}:
\begin{itemize}
\item 4 user manuals (1 per lab), minimum 15 pages each
\item 4 assessment rubrics with quantifiable criteria
\item 1 teacher guide with deployment instructions
\item Initial setup time per teacher $\leq$ 1 hour
\end{itemize}

\textbf{Deliverables}:
\begin{itemize}
\item Student manual (English, markdown/PDF format)
\item Teacher guide with detailed instructions
\item Structured assessment rubrics
\item Technical README with requirements and instructions
\end{itemize}

\subsubsection{OE4: Validate Usability and Educational Effectiveness}

\textbf{Description}: Verify that the package meets educational and technical requirements.

\textbf{Measurable KPIs}:
\begin{itemize}
\item Average resolution time per lab: 90-120 minutes
\item Successful completion rate $\geq$ 85\% (pilot with 10 users)
\item User satisfaction score $\geq$ 4/5
\item Zero critical incidents in production environment
\end{itemize}

\textbf{Deliverables}:
\begin{itemize}
\item Pilot validation report
\item Usability and performance metrics
\item Implemented improvement recommendations
\end{itemize}

%\subsection{Definició del Client i Usuari Final}
\subsection{Definition of the Client and Final User}
%\subsubsection{Client Principal}
\subsection{Principal Client}
\begin{itemize}
\item \textbf{Responsible professor}: Pere Vidiella i Catalan
\item \textbf{Course}: Introduction to Cybersecurity (GEISI, Tecnocampus)
\item \textbf{Context}: Undergraduate teaching in practical cybersecurity
\end{itemize}

\subsubsection{Primary End Users}

\begin{itemize}
\item \textbf{GEISI Students}: 25-30 students per edition
\item \textbf{Profile}: Basic knowledge of networks and operating systems
\item \textbf{Objective}: Learn ethical pentesting in a practical and safe manner
\end{itemize}

\subsubsection{Secondary End Users}

\begin{itemize}
\item \textbf{Cybersecurity Faculty}: Other teachers interested in reusing the material
\item \textbf{Postgraduate Students}: Possible extensions of the material for advanced levels
\end{itemize}

\subsection{Potential Audience}

\subsubsection{Internal Scope (Tecnocampus)}

\begin{itemize}
\item Other courses related to cybersecurity
\item Research projects in computer security
\item Continuing education and professional certifications
\end{itemize}

\subsubsection{External Scope}

\begin{itemize}
\item Educational institutions with training in cybersecurity
\item Specialized vocational training centers
\item Companies with internal training programs in security
\end{itemize}

\subsection{KPIs and Performance Indicators}

\subsubsection{Technical Indicators}

\begin{table}[h!]
\centering
\begin{tabular}{|l|l|l|}
\hline
\textbf{Mètrica} & \textbf{Objectiu} & \textbf{Mètode de Mesura} \\
\hline
Lab deployment time & $\leq$ 2 min & Automated timing \\
\hline
VM boot success rate & $\geq$ 95\% & System logs + validation scripts \\
\hline
Full reset time & $\leq$ 30 seg & Scripts with timestamps \\
\hline
Scenario reproducibility & 100\% & Automated tests \\
\hline
\end{tabular}
\end{table}

\subsubsection{Educational Indicators}

\begin{table}[h!]
\centering
\begin{tabular}{|l|l|l|}
\hline
\textbf{Mètrica} & \textbf{Objectiu} & \textbf{Mètode de Mesura} \\
\hline
Lab resolution time & 90-120 min & Time tracking during pilot \\
\hline
Successful completion rate & $\geq$ 85\% & User progress tracking during pilot \\
\hline
User satisfaction & $\geq$ 4/5 & Post-use survey \\
\hline
Key concept understanding & $\geq$ 80\% correcte & Assessment questionnaire \\
\hline
\end{tabular}
\end{table}

\subsubsection{Quality Indicators}

\begin{table}[h!]
\centering
\begin{tabular}{|l|l|l|}
\hline
\textbf{Mètrica} & \textbf{Objectiu} & \textbf{Mètode de Mesura} \\
\hline
Documentation coverage & 100\% of features & Verification checklist \\
\hline
Critical errors & 0 & Exhaustive testing \\
\hline
EVE-NG version compatibility & 100\% & Tests in multiple environments \\
\hline
Interface usability & $\geq$ 4/5 & UX heuristic evaluation \\
\hline
\end{tabular}
\end{table}

\noindent \textbf{Note}: All KPIs will be measured during the pilot validation phase (April-May 2026) and documented in the final TFG report.


% ============================================================================
% CAPÍTOL 5: METODOLOGIA
% ============================================================================


\section{Methodology}

\subsection{Final Degree Project (TFG) Implementation Process}

% Section pending development

\subsection{Stages, Tasks and Milestones}

% Section pending development

\subsection{Information Search Strategies}

% Section pending development

\subsection{Product Development Methodology}

% Section pending development


% ============================================================================
% CAPÍTOL 6: REQUERIMENTS
% ============================================================================


\section{Functional Requirements}

\subsection{RF1: Design EVE-NG Topologies}

- Create 4 .unl files compatible with EVE-NG 
- Implement OWASP  and CIS  security principles

\subsection{RF2: Implement Vulnerabilities}

- Lab 1: Reconnaissance tools per PTES 
- Lab 2: OWASP Top 10  vulnerabilities
- Lab 3: Network analysis per NIST standards 
- Lab 4: Privilege escalation per CEH framework 

\subsection{RF3: Create Assessment Materials}

- Assessment aligned with GEISI learning outcomes 
- Rubrics based on CIS Controls 

\section{Technological Requirements}

\subsection{TR1: Virtualization}

- EVE-NG Platform 
- KVM Hypervisor
- Network simulation capabilities

\subsection{TR2: Tools}

- Kali Linux  (attacking platform)
- Metasploit Framework  (exploitation)
- Nmap  (reconnaissance)
- Wireshark  (network analysis)
- Burp Suite  (web testing)
- OWASP ZAP  (web scanning)

\subsection{TR3: Scripts}

- Bash scripting
- Python automation

\section{Non-Functional Requirements}

\subsection{NFR1: Performance}

- Deployment per EVE-NG specifications 
- Boot times optimized for lab workflow

\subsection{NFR2: Compliance}

- NIST principles 
- CIS Controls 
- OWASP methodologies 

\subsection{NFR3: Institutional Sustainability}

- Alignment with GEISI curriculum 
- Open-source tools to ensure long-term maintainability



% ============================================================================
% CAPÍTOL 7: ESTUDI DE VIABILITAT
% ============================================================================


%\section{Estudi de la Viabilitat del Projecte}
\section{Feasibility Study of the Project}

%\subsection{Planificació Initial}
\subsection{Initial Planning}

%\subsubsection{Definició de Tasques}
\subsubsection{Task Definition}

% Secció pendent de desenvolupament

%\subsubsection{Diagrama de Gantt}
\subsubsection{Gantt Chart}

% Secció pendent de desenvolupament

%\subsubsection{Camins Crítics}
\subsubsection{Critical Paths}

% Secció pendent de desenvolupament

%\subsection{Pressupost}
\subsection{Budget}

%\subsubsection{Cost de Recursos Humans}
\subsubsection{Cost of Human Resources}

% Secció pendent de desenvolupament

%\subsubsection{Cost de Recursos Hardware i Software}
\subsubsection{Cost of Hardware and Software Resources}

% Secció pendent de desenvolupament

%\subsubsection{Altres Costos}
\subsubsection{Other Costs}

% Secció pendent de desenvolupament

\subsection{Feasibility Analysis}

\subsubsection{Technical Feasibility}

% Secció pendent de desenvolupament

\subsubsection{Economic Feasibility}

% Secció pendent de desenvolupament

\subsubsection{Environmental Feasibility}


% Secció pendent de desenvolupament

\subsubsection{Legal Aspects}

% Secció pendent de desenvolupament

\subsubsection{Gender and Diversity Perspective}

% Secció pendent de desenvolupament


% ============================================================================
% CAPÍTOL 8: DESENVOLUPAMENT I IMPLEMENTACIÓ (PLACEHOLDER)
% ============================================================================


\chapter{Development and Implementation}


\section{Arquitectura General}


\textit{En aquesta secció es descriurà l'arquitectura dels laboratoris, la configuració 
de les xarxes virtuals, i la implementació de les topologies EVE-NG.}


\section{Laboratori 1: Reconeixement i Enumeració}


\subsection{Objectius de Laboratori}


\begin{itemize}
    \item Familiarització amb eines de scanning (nmap, Nessus)
    \item Enumeració de serveis i vulnerabilitats
    \item Documentació de descobriments
\end{itemize}


\subsection{Topologia de Xarxa}


\textit{Diagrama de la xarxa i màquines virtuals que componen aquest laboratori.}


\section{Laboratori 2: Vulnerabilitats d'Aplicacions Web}


\subsection{OWASP Top 10}


\textit{Exploració de les vulnerabilitats més crítiques segons OWASP.}


\section{Laboratori 3: Anàlisi de Tràfic i Criptografia}


\textit{Anàlisi de comunicacions de xarxa amb Wireshark i tècniques de criptografia.}


\section{Laboratori 4: Escalada de Privilegis}


\textit{Tècniques d'escalada de privilegis a sistemes Unix i Windows.}


% ============================================================================
% CAPÍTOL 9: RESULTATS I CONCLUSIONS
% ============================================================================


\chapter{Results and Conclusions}


\section{Lliurables Entregats}


\begin{itemize}
    \item 4 topologies EVE-NG funcionals (.unl)
    \item 8-12 imatges de màquines virtuals optimitzades
    \item Scripts d'automatització per a desplegament
    \item Documentació tècnica completa per laboratori
    \item Guies pràctiques per a estudiants
\end{itemize}


\section{Conclusions}


El desenvolupament d'aquests laboratoris proporciona una base sòlida per a la formació 
en pentesting ètic dins del Grau en Enginyeria Informàtica.


\section{Recomanacions per a Treballs Futurs}


\begin{itemize}
    \item Integració amb altres plataformes educatives (Moodle)
    \item Ampliació amb laboratoris d'IoT i xarxes 5G
    \item Automatització de correccions i qualificacions
\end{itemize}


% ============================================================================
% ANEXOS
% ============================================================================


\appendix


\chapter{Appendix A: Reproducible LaTeX Build Infrastructure}


\section{Configuració de Docker i TeX Live}


La memòria del present TFG s'ha desenvolupat usant una infraestructura de build reproducible 
basada en TeX Live 2023, latexmk i Docker Compose. Aquesta aproximació garanteix que tots 
els artefactes (PDF, índexs, glosaris) es generin de manera consistent.


\subsection{Hores Invertides en Infraestructura}


\begin{table}[H]
    \centering
    \begin{tabular}{|l|l|r|}
    \hline
    \textbf{Tasca} & \textbf{Descripció} & \textbf{Hores} \\
    \hline
    TeX Live + Docker & Setup de compilació & 2.5h \\
    latexmk + .latexmkrc & Build automatitzat & 1.5h \\
    Estructura modular (chapters/) & Organització de capítols & 3.5h \\
    Validació i testing & Compilació final & 0.5h \\
    \hline
    \textbf{SUBTOTAL} & & \textbf{8.0h} \\
    \hline
    \end{tabular}
\end{table}


\chapter{Appendix B: Laboratory Infrastructure (Homelab)}


\section{Especificacions del Homelab}


\begin{table}[H]
    \centering
    \begin{tabular}{|l|l|r|}
    \hline
    \textbf{Component} & \textbf{Especificació} & \textbf{Hores Setup} \\
    \hline
    Host VMware/Hyper-V & CPU 8-core, RAM 32GB, SSD 500GB & 3.0h \\
    EVE-NG & Simulador de xarxes corporatiu & 4.0h \\
    Kali Linux x3 & Màquines de pentesting & 2.5h \\
    Windows Server & Sistema objectiu & 1.5h \\
    Xarxes virtuals & 3 VLANS, DMZ, backend & 2.0h \\
    \hline
    \textbf{SUBTOTAL} & & \textbf{13.0h} \\
    \hline
    \end{tabular}
\end{table}


\chapter{Appendix C: Total Time Tracking}


\begin{table}[H]
    \centering
    \begin{tabular}{|l|r|r|}
    \hline
    \textbf{Activitat} & \textbf{Hores} & \textbf{Percentatge} \\
    \hline
    Infraestructura LaTeX & 8.0h & 6\% \\
    Infraestructura Homelab & 13.0h & 10\% \\
    Investigació i Disseny & 20.0h & 15\% \\
    Implementació de Laboratoris & 50.0h & 37\% \\
    Documentació & 25.0h & 19\% \\
    Testing i Validació & 15.0h & 11\% \\
    Gestió de Projecte & 5.0h & 4\% \\
    \hline
    \textbf{TOTAL ESTIMAT} & \textbf{136.0h} & \textbf{100\%} \\
    \hline
    \end{tabular}
\end{table}


% ============================================================================
% BIBLIOGRAFIA
% ============================================================================


\cleardoublepage
\phantomsection
\addcontentsline{toc}{chapter}{Bibliography}
\printbibliography


% Incluir el appendici de registre de temps si existeix (evita fallades si falta)
%\subsubsection{Week X (XX-YY Setembre/Octubre/Novembre 2025) - [TÍTOL]}
%
%\begin{table}[H]
%\centering
%\small
%\begin{tabularx}{\textwidth}{|l|l|r|l|l|}
%\hline
%\textbf{Date} & \textbf{Activity} & \textbf{Hours} & \textbf{Task IDs} & \textbf{Memory Section(s)} \\
%\hline
%Nov 23 & [ACTIVITAT AQUÍ] & 2.5h & 1.2 & \S 1.4 (State of the Art) \\
%\hline
%\multicolumn{3}{|r|}{\textbf{Week X Subtotal}} & & \textbf{XXh} \\
%\hline
%\end{tabularx}
%\caption{Week X Time Log: [TÍTOL]}
%\label{tab:weekX_log}
%\end{table}
%
%\paragraph{Impact Analysis - Week X:}
%\begin{itemize}
%    \item \textbf{Directly Billable}: XXh -> Task 1.X
%    \item \textbf{Section Mapping}: \S 1.X: +Xh
%    \item \textbf{Status}: In Progress / Complete
%\end{itemize}
%
%---


\section{A. Time Log and Project Tracking}

\subsection{A.1 Weekly Time Entry Log (October-November 2025)}

This appendix documents all recorded hours with detailed tracking of activities, task allocation, and memory section impact. Hours are classified as either directly billable (mapped to specific TFG sections) or supportive (infrastructure, general management).

\subsubsection{Week 1 (1-8 October 2025) - Project Startup}

\begin{table}[H]
\centering
\small
\begin{tabularx}{\textwidth}{|l|l|r|l|l|}
\hline
\textbf{Date} & \textbf{Activity} & \textbf{Hours} & \textbf{Task IDs} & \textbf{Memory Section(s)} \\
\hline
Oct 1-3 & Project initialization + TFG regulations review & 2.0h & 0.1, 1.0 & \S 1.3 (Context) \\
\hline
Oct 4-5 & Initial tutor contact + planning & 2.0h & 1.0 & \S 1.3 (Objectives) \\
\hline
Oct 6-8 & Normativa analysis + compliance check & 3.0h & 1.0 & \S 1.3 (Regulatory Framework) \\
\hline
\multicolumn{3}{|r|}{\textbf{Week 1 Subtotal}} & & \textbf{7.0h} \\
\hline
\end{tabularx}
\caption{Week 1 Time Log: Project Startup and Planning}
\label{tab:week1_log}
\end{table}

\paragraph{Impact Analysis - Week 1:}
\begin{itemize}
    \item \textbf{Directly Billable}: 7.0h -> Task 1.0 (Thesis Development baseline)
    \item \textbf{Section Mapping}: 
    \begin{itemize}
        \item \S 1.3 Objective of TFG: +3.0h (context establishment)
        \item Appendix: +4.0h (regulatory framework, project governance)
    \end{itemize}
    \item \textbf{Status}: Foundational work; establishes project scope and constraints
\end{itemize}

---

\subsubsection{Week 2 (9-15 October 2025) - Technical Research}

\begin{table}[H]
\centering
\small
\begin{tabularx}{\textwidth}{|l|l|r|l|l|}
\hline
\textbf{Date} & \textbf{Activity} & \textbf{Hours} & \textbf{Task IDs} & \textbf{Memory Section(s)} \\
\hline
Oct 9-10 & EVE-NG platform research + alternatives analysis & 3.5h & 1.2 & \S 1.4 (State of the Art) \\
\hline
Oct 11-12 & Educational context analysis (OWASP, NIST frameworks) & 3.5h & 1.2 & \S 1.4 (State of the Art) \\
\hline
Oct 13 & Cybersecurity education models research & 2.0h & 1.2 & \S 1.4 (Existing Solutions) \\
\hline
Oct 14-15 & Home Lab context analysis + infrastructure options & 1.5h & 0.1 & \S 1.8 (Infrastructure Section) \\
\hline
\multicolumn{3}{|r|}{\textbf{Week 2 Subtotal}} & & \textbf{10.5h} \\
\hline
\end{tabularx}
\caption{Week 2 Time Log: Technical Research and Context Analysis}
\label{tab:week2_log}
\end{table}

\paragraph{Impact Analysis - Week 2:}
\begin{itemize}
    \item \textbf{Directly Billable}: 10.5h -> Task 1.2 (State of the Art - Phase 1)
    \item \textbf{Section Mapping}:
    \begin{itemize}
        \item \S 1.4 State of the Art: +10.0h (EVE-NG alternatives, educational frameworks, existing solutions)
        \item \S 1.8 Resource Inventory: +0.5h (infrastructure assessment)
    \end{itemize}
    \item \textbf{Status}: Core research phase; foundation for design decisions
    \item \textbf{Note}: 75\% of content for Section 1.4 now available
\end{itemize}

---

\subsubsection{Week 3 (16-22 October 2025) - Objectives and Design}

\begin{table}[H]
\centering
\small
\begin{tabularx}{\textwidth}{|l|l|r|l|l|}
\hline
\textbf{Date} & \textbf{Activity} & \textbf{Hours} & \textbf{Task IDs} & \textbf{Memory Section(s)} \\
\hline
Oct 16-17 & SMART objectives formulation with KPIs & 4.0h & 1.1 & \S 1.5 (Objectives and Scope) \\
\hline
Oct 18-19 & Pentesting tools ecosystem research & 4.0h & 1.2 & \S 1.4 (Existing Tools) \\
\hline
Oct 20 & Design approach definition (4 labs structure) & 3.0h & 1.5 & \S 1.5 (Design Overview) \\
\hline
Oct 21-22 & Requirements specification (functional + technical) & 3.5h & 1.3 & \S 1.7 (Requirements Definition) \\
\hline
Oct 22 & Lab topology design sketches & 2.5h & 1.5 & \S 1.5 (Design Justification) \\
\hline
\multicolumn{3}{|r|}{\textbf{Week 3 Subtotal}} & & \textbf{17.0h} \\
\hline
\end{tabularx}
\caption{Week 3 Time Log: SMART Objectives and Technical Design}
\label{tab:week3_log}
\end{table}

\paragraph{Impact Analysis - Week 3:}
\begin{itemize}
    \item \textbf{Directly Billable}: 17.0h -> Tasks 1.1, 1.2, 1.3, 1.5
    \item \textbf{Section Mapping}:
    \begin{itemize}
        \item \S 1.5 Objectives and Scope: +7.0h (SMART formulation, design overview)
        \item \S 1.4 State of the Art: +4.0h (tools research completing section)
        \item \S 1.7 Requirements Definition: +3.5h (functional specifications)
        \item \S 1.5 Design Justification: +2.5h (topology design rationale)
    \end{itemize}
    \item \textbf{Status}: Completion of Phase 1 analysis framework
    \item \textbf{Critical Output}: 4 SMART objectives with 16 measurable KPIs
\end{itemize}

---

\subsubsection{Week 4 (23-26 October 2025) - Preliminary Project Writing}

\begin{table}[H]
\centering
\small
\begin{tabularx}{\textwidth}{|l|l|r|l|l|}
\hline
\textbf{Date} & \textbf{Activity} & \textbf{Hours} & \textbf{Task IDs} & \textbf{Memory Section(s)} \\
\hline
Oct 23-24 & Avantprojecte: Object section redaction & 4.0h & 1.0, Avant & \S 1.3 (Objective + Context) \\
\hline
Oct 24-25 & Avantprojecte: Objectives + Scope section & 5.0h & 1.1, Avant & \S 1.5 (Objectives + Scope) \\
\hline
Oct 25-26 & Administrative docs + deliverable preparation & 2.0h & 1.0 & \S 1.3, Appendix A \\
\hline
Oct 26 & Bibliography collection and organization & 2.0h & 1.0 & Appendix: References \\
\hline
\multicolumn{3}{|r|}{\textbf{Week 4 Subtotal}} & & \textbf{13.0h} \\
\hline
\end{tabularx}
\caption{Week 4 Time Log: Preliminary Project (Avantprojecte) Compilation}
\label{tab:week4_log}
\end{table}

\paragraph{Impact Analysis - Week 4:}
\begin{itemize}
    \item \textbf{Directly Billable}: 13.0h -> Avantprojecte milestone
    \item \textbf{Section Mapping}:
    \begin{itemize}
        \item \S 1.3 Objective of TFG: +4.0h (comprehensive framing)
        \item \S 1.5 Objectives and Scope: +5.0h (detailed objectives + scope)
        \item General: +4.0h (bibliography, administrative setup)
    \end{itemize}
    \item \textbf{Status}: First major milestone (Avantprojecte) preparation
    \item \textbf{Note}: Preliminary Project submission ready for tutor review
\end{itemize}

---

\subsubsection{Technical Setup (September-October) - Parallel Track}

\begin{table}[H]
\centering
\small
\begin{tabularx}{\textwidth}{|l|l|r|l|l|}
\hline
\textbf{Date} & \textbf{Activity} & \textbf{Hours} & \textbf{Task IDs} & \textbf{Memory Section(s)} \\
\hline
Sep 1-15 & Home Lab initialization (EVE-NG + infrastructure) & 8.0h & 0.1 & \S 1.8 (Infrastructure), Appendix B \\
\hline
Sep 15-30 & Kali Linux + target systems setup & 3.5h & 0.3 & \S 1.6 (Methodology), Appendix B \\
\hline
Oct 1-6 & LaTeX environment setup (Overleaf + local Docker) & 2.0h & 0.2 & Appendix C (Technical Setup) \\
\hline
Oct 6-12 & LaTeX build configuration (.latexmkrc) & 1.0h & 0.2 & Appendix C (Build Process) \\
\hline
Oct 6-15 & Bibliography + Glossary setup (IEEE format) & 1.5h & 0.2 & Appendix C (Bibliography) \\
\hline
Oct 15-31 & Home Lab infrastructure completion + documentation & 10.0h & 0.4 & \S 1.8 (As-Built Docs), Appendix B \\
\hline
\multicolumn{3}{|r|}{\textbf{Technical Setup Subtotal}} & & \textbf{26.0h} \\
\hline
\end{tabularx}
\caption{Technical Setup Hours (Parallel to Weeks 1-4)}
\label{tab:technical_setup}
\end{table}

\paragraph{Impact Analysis - Technical Setup:}
\begin{itemize}
    \item \textbf{Directly Billable}: 26.0h -> Tasks 0.1, 0.2, 0.3, 0.4
    \item \textbf{Section Mapping}:
    \begin{itemize}
        \item \S 1.8 Resource Inventory: +18.0h (infrastructure, as-built documentation)
        \item \S 1.6 Methodology: +3.5h (Home Lab methodology)
        \item Appendix B (Infrastructure): +14.5h (detailed technical setup)
        \item Appendix C (LaTeX Setup): +4.5h (documentation environment)
    \end{itemize}
    \item \textbf{Status}: Foundation for all subsequent development
    \item \textbf{Note}: These hours establish reusable infrastructure; cost amortized across all phases
\end{itemize}

---

\subsubsection{Tutor Meetings and Feedback Cycles}

\begin{table}[H]
\centering
\small
\begin{tabularx}{\textwidth}{|l|l|r|l|l|}
\hline
\textbf{Date} & \textbf{Activity} & \textbf{Hours} & \textbf{Task IDs} & \textbf{Memory Section(s)} \\
\hline
Oct 27 & Tutor Meeting 1: Process definition + platform discussion & 1.5h & Mgmt & Appendix A (Project Governance) \\
\hline
Nov 4 & Tutor Meeting 2: Format + materials clarification & 1.5h & Mgmt & Appendix A (Meeting Notes) \\
\hline
\multicolumn{3}{|r|}{\textbf{Tutor Meetings Subtotal}} & & \textbf{3.0h} \\
\hline
\end{tabularx}
\caption{Tutor Meetings and Management Hours}
\label{tab:tutor_meetings}
\end{table}

---

\subsubsection{November Intensive Work Session (22 November 2025)}

\begin{table}[H]
\centering
\small
\begin{tabularx}{\textwidth}{|l|l|r|l|l|}
\hline
\textbf{Date} & \textbf{Activity} & \textbf{Hours} & \textbf{Task IDs} & \textbf{Memory Section(s)} \\
\hline
Nov 22 & Pentesting Methodology document (comprehensive) & 3.0h & 1.4 & \S 1.6 (Methodology) \\
\hline
Nov 22 & Design Justification framework (8 major decisions) & 2.0h & 1.5 & \S 1.5 (Design Justification) \\
\hline
Nov 22 & Gantt chart translation + critical path analysis & 1.5h & 1.6 & \S 1.8 (Planning) \\
\hline
Nov 22 & Resource Inventory framework + time tracking system & 1.5h & 1.8 & \S 1.8 (Resources) \\
\hline
\multicolumn{3}{|r|}{\textbf{Nov 22 Session Subtotal}} & & \textbf{8.0h} \\
\hline
\end{tabularx}
\caption{November 22 Intensive AI-Assisted Working Session}
\label{tab:nov22_session}
\end{table}

\paragraph{Impact Analysis - Nov 22:}
\begin{itemize}
    \item \textbf{Directly Billable}: 8.0h -> Tasks 1.4, 1.5, 1.6, 1.8
    \item \textbf{Section Mapping}:
    \begin{itemize}
        \item \S 1.6 Methodology: +3.0h (pentesting framework)
        \item \S 1.5 Design Justification: +2.0h (architectural decisions)
        \item \S 1.8 Resource Inventory: +3.0h (planning + tracking)
    \end{itemize}
    \item \textbf{Status}: Critical Phase 1 acceleration
    \item \textbf{Output}: 4 formal academic documents, 1 visual Gantt, planning framework
\end{itemize}

---

\subsection{A.2 Cumulative Impact Analysis by Memory Section}

\begin{table}[H]
\centering
\small
\begin{tabularx}{\textwidth}{|l|r|r|l|}
\hline
\textbf{Memory Section} & \textbf{Direct Hours} & \textbf{Estimated \% Complete} & \textbf{Status} \\
\hline
\S 1.3 Objective of TFG & 7.0h & 90\% & Content draft ready \\
\hline
\S 1.4 State of the Art & 14.0h & 75\% & Research phase complete, writing pending \\
\hline
\S 1.5 Objectives and Scope & 12.0h & 95\% & Nearly complete (SMART KPIs fully defined) \\
\hline
\S 1.6 Methodology & 6.5h & 80\% & Framework ready, integration pending \\
\hline
\S 1.7 Requirements Definition & 3.5h & 50\% & Functional specs drafted, technical specs pending \\
\hline
\S 1.8 Resource Inventory & 21.0h & 60\% & Framework complete, details accumulating \\
\hline
Appendix A (Project Governance) & 4.0h & 80\% & Structure set, tutor meeting notes integrated \\
\hline
Appendix B (Infrastructure) & 14.5h & 85\% & Home Lab as-built documentation \\
\hline
Appendix C (LaTeX Setup) & 4.5h & 90\% & Build process documented \\
\hline
Bibliography (IEEE Format) & 2.0h & 40\% & Initial structure, full integration pending \\
\hline
\multicolumn{2}{|r|}{\textbf{Total through Nov 22}} & \textbf{89.0h} & \\
\hline
\end{tabularx}
\caption{Cumulative Impact by Memory Section}
\label{tab:impact_by_section}
\end{table}

---

\subsection{A.3 Shared Hours: Cross-Section Impact Mapping}

Several activities impact \textbf{multiple memory sections}. The following table maps shared hours:

\begin{table}[H]
\centering
\small
\begin{tabularx}{\textwidth}{|l|r|p{2cm}|p{2cm}|}
\hline
\textbf{Activity} & \textbf{Hours} & \textbf{Primary Section} & \textbf{Secondary Sections} \\
\hline
EVE-NG Research (Oct 9-12) & 3.5h & \S 1.4 (State of Art) & \S 1.5 (Design), \S 1.8 (Infrastructure) \\
\hline
Pentesting Frameworks (Oct 11-13) & 2.0h & \S 1.4 (State of Art) & \S 1.6 (Methodology) \\
\hline
Home Lab Setup (Sep 15 - Oct 31) & 13.5h & \S 1.8 (Resources) & Appendix B, \S 1.6 (Methodology) \\
\hline
SMART Objectives (Oct 16-17) & 4.0h & \S 1.5 (Objectives) & \S 1.3 (Context), \S 1.8 (Planning) \\
\hline
Design Decisions (Oct 20-22) & 5.5h & \S 1.5 (Design) & \S 1.7 (Requirements), \S 1.6 (Methodology) \\
\hline
\end{tabularx}
\caption{Shared Hours: Cross-Section Impact}
\label{tab:shared_hours}
\end{table}

\paragraph{Allocation Strategy for Shared Hours:}

When hours benefit multiple sections, allocation follows this priority:
\begin{enumerate}
    \item \textbf{Primary allocation}: The section where most substantial content is generated
    \item \textbf{Secondary credit}: 25-50\% of hours attributed to supporting sections
    \item \textbf{Calculation}: Primary section receives 100\% count; secondary sections receive 25-50\% (noted as ``shared'')
\end{enumerate}

\textbf{Example}: 3.5h of EVE-NG Research:
\begin{itemize}
    \item \S 1.4 State of the Art: +3.5h (100\%) — primary section
    \item \S 1.5 Design: +0.875h (25\%) — shared contribution
    \item \S 1.8 Infrastructure: +0.875h (25\%) — shared contribution
\end{itemize}

---

\subsection{A.4 Pending Hours and Uncertainty Matrix}

Several tasks remain in-progress or not yet started. The following table estimates hours with uncertainty ranges:

\begin{table}[H]
\centering
\small
\begin{tabularx}{\textwidth}{|l|r|r|r|l|}
\hline
\textbf{Task} & \textbf{Planned} & \textbf{Minimum} & \textbf{Maximum} & \textbf{Status} \\
\hline
1.2 State of the Art (completion) & 15h & 3h & 7h & 75\% done, ~5h remaining \\
\hline
1.3 Functional Requirements (polish) & 10h & 2h & 4h & 80\% done, ~2h remaining \\
\hline
1.7 Risk Analysis \& Plan B & 15h & 12h & 18h & Not started; high uncertainty \\
\hline
1.8 Resource Inventory (finalization) & 5h & 2h & 3h & 60\% done, ~1.5h remaining \\
\hline
Hackathon Investigation (parallel) & 25h & 8h & 12h & 40\% done, ~5-7h remaining \\
\hline
Bibliography Completion & 5h & 3h & 6h & 40\% done, ~3h remaining \\
\hline
\end{tabularx}
\caption{Pending Hours with Uncertainty Ranges}
\label{tab:pending_hours}
\end{table}

\paragraph{Interpretation of Ranges:}
\begin{itemize}
    \item \textbf{Minimum}: Best-case scenario (good momentum, no blockers)
    \item \textbf{Maximum}: Worst-case scenario (additional research needed, iterations required)
    \item \textbf{Planned}: Original Gantt estimate
\end{itemize}

---

\subsection{A.5 Running Hour Total and Projections}

\begin{table}[H]
\centering
\small
\begin{tabularx}{\textwidth}{|l|r|r|r|}
\hline
\textbf{Period} & \textbf{Hours Logged} & \textbf{Cumulative} & \textbf{\% of Phase 1} \\
\hline
October 1-8 (Week 1) & 7.0h & 7.0h & 7.6\% \\
\hline
October 9-15 (Week 2) & 10.5h & 17.5h & 19.0\% \\
\hline
October 16-22 (Week 3) & 17.0h & 34.5h & 37.5\% \\
\hline
October 23-26 (Week 4) & 13.0h & 47.5h & 51.6\% \\
\hline
Technical Setup (parallel) & 26.0h & 73.5h & 79.8\% \\
\hline
Tutor Meetings & 3.0h & 76.5h & 83.1\% \\
\hline
November 22 Session & 8.0h & 84.5h & 91.8\% \\
\hline
\multicolumn{2}{|r|}{\textbf{Phase 1 Target}} & \textbf{92h} & \textbf{100\%} \\
\hline
\multicolumn{2}{|r|}{\textbf{Remaining to Phase 1 Close}} & \textbf{~7.5h} & \textbf{8.2\%} \\
\hline
\end{tabularx}
\caption{Running Hour Total through Phase 1}
\label{tab:running_total}
\end{table}

\paragraph{Projection to Phase 1 Completion (Dec 20, 2025):}

Based on current velocity:
\begin{itemize}
    \item \textbf{Current total}: 84.5h logged (91.8\% of Phase 1 planned hours)
    \item \textbf{Remaining}: ~7.5h to reach 92h target
    \item \textbf{Pending tasks}: 1.7 (Risk Analysis), final polish on 1.2, 1.3, 1.8
    \item \textbf{Projection}: Phase 1 completion achievable by Dec 15 with focused effort
    \item \textbf{Buffer}: 5 days for revision and tutor feedback incorporation
\end{itemize}

---

\subsection{A.6 Key Findings and Recommendations}

\subsubsection{Hours Well-Invested}

\begin{itemize}
    \item \textbf{Foundations (Setup)}: 26h on technical infrastructure provides reusable foundation for entire project
    \item \textbf{Planning (Analysis)}: 47.5h on Phase 1 analysis creates solid blueprint for development phases
    \item \textbf{Early engagement with tutor}: 3h in Oct meetings prevented scope creep and format issues
\end{itemize}

\subsubsection{High-Impact Emerging Tasks}

\begin{itemize}
    \item \textbf{State of the Art (1.4)}: Currently 14h invested; likely to expand to 18-20h with comprehensive literature review
    \item \textbf{Risk Analysis (1.7)}: Not yet started; estimated 15-18h; critical for viability assessment
    \item \textbf{Bibliography}: Currently 2h; needs 3-5h more for comprehensive IEEE-formatted references
\end{itemize}

\subsubsection{Shared Hour Accounting}

\begin{itemize}
    \item \textbf{Multiple benefits}: Many activities (especially research and planning) benefit 2-3 memory sections simultaneously
    \item \textbf{Conservative accounting}: Primary allocation gives 100\% to most impactful section; secondary sections receive 25-50\%
    \item \textbf{Total billable value}: 84.5h logged creates approximately 95-105 hours of effective work across sections when accounting for shared benefits
\end{itemize}

---

\subsection{A.7 Appendix: Month-by-Month Summary Table}

\begin{table}[H]
\centering
\small
\begin{tabularx}{\textwidth}{|l|r|l|l|}
\hline
\textbf{Month} & \textbf{Hours} & \textbf{Main Activities} & \textbf{Deliverables} \\
\hline
September 2025 & ~15h & Home Lab setup, infrastructure prep & Initial EVE-NG environment \\
\hline
October 2025 & ~47h & Planning, research, Avantprojecte prep & SMART objectives, Preliminary Project draft \\
\hline
November 1-21, 2025 & ~15h & Technical setup completion, prep work & Documentation, infrastructure finalization \\
\hline
November 22, 2025 & 8h & Intensive documentation sprint & 4 formal documents, Gantt chart, tracking system \\
\hline
\multicolumn{2}{|r|}{\textbf{Total through Nov 22}} & & \textbf{~85h} \\
\hline
\end{tabularx}
\caption{Month-by-Month Summary}
\label{tab:monthly_summary}
\end{table}

---

\paragraph{Conclusion:} The time log demonstrates systematic progress through Phase 1 analysis with strong documentation of activities, clear impact mapping to memory sections, and realistic projections for remaining work. The detailed tracking enables transparency and supports project management decisions through completion.




\end{document}
